\documentclass{amsart}
\usepackage{calc,amssymb,amsthm,amsmath,setspace}
\RequirePackage[dvipsnames,usenames]{xcolor}

\usepackage{fancyvrb}
\usepackage{hyperref}
\hypersetup{
bookmarks,
bookmarksdepth=3,
bookmarksopen,
bookmarksnumbered,
pdfstartview=FitH,
colorlinks,backref,hyperindex,
linkcolor=Sepia,
anchorcolor=BurntOrange,
citecolor=MidnightBlue,
citecolor=OliveGreen,
filecolor=BlueViolet,
menucolor=Yellow,
urlcolor=OliveGreen
}
\usepackage{xspace}
%\usepackage{rotating}
\interfootnotelinepenalty=100000

\usepackage{mabliautoref}
\usepackage{colonequals}
\frenchspacing
%%   Setup to use Ralph Smith Formal Script font:
%%   Provided by Sandor Kovacs to Karl Schwede, modified further
\DeclareFontFamily{OMS}{rsfs}{\skewchar\font'60}
\DeclareFontShape{OMS}{rsfs}{m}{n}{<-5>rsfs5 <5-7>rsfs7 <7->rsfs10 }{}
\DeclareSymbolFont{rsfs}{OMS}{rsfs}{m}{n}
\DeclareSymbolFontAlphabet{\scr}{rsfs}

% \newenvironment{pf}[1][{\it Proof:}]{\begin{trivlist}
% \item[\hskip \labelsep {\bfseries #1}]}{\end{trivlist}}

%%***Collaborative editing commands***
\newcommand{\note}[1]{\marginpar{\sffamily\tiny #1}}

\def\mytodo#1{\textcolor{Mahogany}%
{\sffamily\footnotesize\newline{\color{Mahogany}\fbox{\parbox{\textwidth-15pt}{#1}}}\newline}}


%%***Common Names***
\newcommand{\Cech}{{$\check{\text{C}}$ech} }
\newcommand{\mustata}{Musta{\c{t}}{\u{a}}}
\newcommand{\etale}{{\'e}tale }


%%%************Functors, derived categories and related****************
\newcommand{\myR}{{\bf R}}
\newcommand{\myH}{{\bf h}}
\newcommand{\myL}{{\bf L}}
\newcommand{\qis}{\simeq_{\text{qis}}}
\newcommand{\tensor}{\otimes}
\newcommand{\hypertensor}{{\uuline \tensor}}
\newcommand{\mydot}{{{\,\begin{picture}(1,1)(-1,-2)\circle*{2}\end{picture}\ }}}
\newcommand{\blank}{\underline{\hskip 10pt}}
\DeclareMathOperator{\trunc}{{trunc}}

%%*************Diagram of scheme notation
\DeclareMathOperator{\kSch}{{Sch}}
\DeclareMathOperator{\kCat}{{Cat}}
\DeclareMathOperator{\kHrc}{{Hrc}}
\newcommand{\uSch}{\underline{\kSch}}
\newcommand{\uCat}{\underline{\kCat}}
\newcommand{\uHrc}{\underline{\kHrc}}
\newcommand{\Ho}{\text{Ho}}

%%%*************Du Bois short hand******************
\newcommand{\DuBois}[1]{{\uuline \Omega {}^0_{#1}}}
\newcommand{\FullDuBois}[1]{{\uuline \Omega {}^{\mydot}_{#1}}}

\newcommand{\modGens}[2]{{{\bf \mu}_{#1}{\left(#2\right)}}}


%%%*************F-singularity short hand************
\newcommand{\tauCohomology}{T}
\newcommand{\FCohomology}{S}
\newcommand{\Ram}{\mathrm{Ram}}
\DeclareMathOperator{\Tr}{Tr}
\newcommand{\roundup}[1]{\lceil #1 \rceil}
\newcommand{\rounddown}[1]{\lfloor #1 \rfloor}
\newcommand{\HK}{\mathrm{HK}}

%%***************Latin short hand*******************
\newcommand{\cf}{{\itshape cf.} }
\newcommand{\loccit}{{\itshape loc. cit.} }
\newcommand{\ie}{{\itshape i.e.} }

%%************Script**********************
\newcommand{\sA}{\scr{A}}
\newcommand{\sB}{\scr{B}}
\newcommand{\sC}{\scr{C}}
\newcommand{\sD}{\scr{D}}
\newcommand{\sE}{\scr{E}}
\newcommand{\sF}{\scr{F}}
\newcommand{\sG}{\scr{G}}
\newcommand{\sH}{\scr{H}}
\newcommand{\sI}{\scr{I}}
\newcommand{\sJ}{\scr{J}}
\newcommand{\sK}{\scr{K}}
\newcommand{\sL}{\scr{L}}
\newcommand{\sM}{\scr{M}}
\newcommand{\sN}{\scr{N}}
\newcommand{\sO}{\scr{O}}
\newcommand{\sP}{\scr{P}}
\newcommand{\sQ}{\scr{Q}}
\newcommand{\sR}{\scr{R}}
\newcommand{\sS}{\scr{S}}
\newcommand{\sT}{\scr{T}}
\newcommand{\sU}{\scr{U}}
\newcommand{\sV}{\scr{V}}
\newcommand{\sW}{\scr{W}}
\newcommand{\sX}{\scr{X}}
\newcommand{\sY}{\scr{Y}}
\newcommand{\sZ}{\scr{Z}}

\newcommand{\enm}[1]{\ensuremath{#1}}
\newcommand{\mJ}{\mathcal{J}}
\newcommand{\uTwo}{\underline{2}}
\newcommand{\uOne}{\underline{1}}
\newcommand{\uZero}{\underline{0}}

\newcommand{\ba}{\mathfrak{a}}
\newcommand{\bb}{\mathfrak{b}}
\newcommand{\bc}{\mathfrak{c}}
\newcommand{\bff}{\mathfrak{f}}
\newcommand{\bm}{\mathfrak{m}}
\newcommand{\bn}{\mathfrak{n}}
\newcommand{\bp}{\mathfrak{p}}
\newcommand{\bq}{\mathfrak{q}}
\newcommand{\bt}{\mathfrak{t}}
\newcommand{\fra}{\mathfrak{a}}
\newcommand{\frb}{\mathfrak{b}}
\newcommand{\frc}{\mathfrak{c}}
\newcommand{\frf}{\mathfrak{f}}
\newcommand{\frm}{\mathfrak{m}}
%\renewcommand{\frm}{\mathfrak{m}}
\newcommand{\frn}{\mathfrak{n}}
\newcommand{\frp}{\mathfrak{p}}
\newcommand{\frq}{\mathfrak{q}}
\newcommand{\frt}{\mathfrak{t}}

\newcommand{\bA}{\mathbb{A}}
\newcommand{\bB}{\mathbb{B}}
\newcommand{\bC}{\mathbb{C}}
\newcommand{\bD}{\mathbb{D}}
\newcommand{\bE}{\mathbb{E}}
\newcommand{\bF}{\mathbb{F}}
\newcommand{\bG}{\mathbb{G}}
\newcommand{\bH}{\mathbb{H}}
\newcommand{\bI}{\mathbb{I}}
\newcommand{\bJ}{\mathbb{J}}
\newcommand{\bK}{\mathbb{K}}
\newcommand{\bL}{\mathbb{L}}
\newcommand{\bM}{\mathbb{M}}
\newcommand{\bN}{\mathbb{N}}
\newcommand{\bO}{\mathbb{O}}
\newcommand{\bP}{\mathbb{P}}
\newcommand{\bQ}{\mathbb{Q}}
\newcommand{\bR}{\mathbb{R}}
\newcommand{\bS}{\mathbb{S}}
\newcommand{\bT}{\mathbb{T}}
\newcommand{\bU}{\mathbb{U}}
\newcommand{\bV}{\mathbb{V}}
\newcommand{\bW}{\mathbb{W}}
\newcommand{\bX}{\mathbb{X}}
\newcommand{\bY}{\mathbb{Y}}
\newcommand{\bZ}{\mathbb{Z}}


\newcommand{\al}{\alpha}
\newcommand{\be}{\beta}
\newcommand{\ga}{\gamma}
\newcommand{\de}{\delta}
\newcommand{\pa}{\partial}   %pretend its Greek
\newcommand{\epz}{\varepsilon}
\newcommand{\ph}{\phi}
\newcommand{\phz}{\varphi}
\newcommand{\et}{\eta}
\newcommand{\io}{\iota}
\newcommand{\ka}{\kappa}
\newcommand{\la}{\lambda}
\newcommand{\tha}{\theta}
\newcommand{\thz}{\vartheta}
\newcommand{\rh}{\rho}
\newcommand{\si}{\sigma}
\newcommand{\ta}{\tau}
\newcommand{\ch}{\chi}
\newcommand{\ps}{\psi}
\newcommand{\ze}{\zeta}
\newcommand{\om}{\omega}
\newcommand{\GA}{\Gamma}
\newcommand{\LA}{\Lambda}
\newcommand{\DE}{\Delta}
\newcommand{\SI}{\Sigma}
\newcommand{\THA}{\Theta}
\newcommand{\OM}{\Omega}
\newcommand{\XI}{\Xi}
\newcommand{\UP}{\Upsilon}
\newcommand{\PI}{\Pi}
\newcommand{\PS}{\Psi}
\newcommand{\PH}{\Phi}

\newcommand{\com}{\circ}     % composition of functions
\newcommand{\iso}{\simeq}    % preferred isomorphism symbol
\newcommand{\ten}{\otimes}   % tensor product
\newcommand{\add}{\oplus}    % direct sum

\newcommand{\ul}{\underline}
\newcommand{\nsubset}{\not\subset}
\newcommand{\tld}{\widetilde }
\renewcommand{\:}{\colon}


\newcommand{\rtarr}{\longrightarrow}
\newcommand{\ltarr}{\longleftarrow}
\newcommand{\from}{\longleftarrow}
\newcommand{\monoto}{\lhook\joinrel\relbar\joinrel\rightarrow}
\newcommand{\epito}{\relbar\joinrel\twoheadrightarrow}

%%%%%%%%% math short hand
%%%% gothic
\newcommand{\Schs}{\mathfrak S\mathfrak c\mathfrak h_{S}}
\newcommand{\LocFrees}{\mathfrak L\mathfrak o\mathfrak c\mathfrak F\mathfrak
 r\mathfrak e\mathfrak e_{S}}
\newcommand{\A}{\mathfrak A}
\newcommand{\Ab}{\mathfrak A\mathfrak b}
\newcommand{\B}{\mathfrak B}
\newcommand{\M}{\mathfrak M\mathfrak o\mathfrak d}
\newcommand{\Mg}{\mathfrak M_g}
\newcommand{\Mgbar}{\overline{\mathfrak M}_g}
\newcommand{\Mh}{\mathfrak M_h}
\newcommand{\Mhbar}{\overline{\mathfrak M}_h}
\newcommand{\maxm}{\mathfrak m}

%%%% curly
\newcommand{\m}{\scr M}
\newcommand{\n}{\scr N}
\newcommand{\cO}{\mathcal O}
\renewcommand{\O}{\mathcal O}
\newcommand{\f}{\scr F}
\renewcommand{\O}{\scr O}
\newcommand{\I}{\scr I}
\newcommand{\J}{\scr{J}}

%%%% Blackboard bold
\newcommand{\C}{\mathbb {C}}
\newcommand{\N}{\mathbb {N}}
\newcommand{\R}{\mathbb {R}}
\newcommand{\PP}{\mathbb {P}}
\newcommand{\Z}{\mathbb {Z}}
\newcommand{\Q}{\mathbb {Q}}
\renewcommand{\r}{\mathbb R^{+}}
\newcommand{\NZ}{\mbox{$\mathbb{N}$}}
\renewcommand{\O}{\mbox{$\mathcal{O}$}}
\renewcommand{\P}{\mathbb{P}}
\newcommand{\ZZ}{\mbox{$\mathbb{Z}$}}
%%%%
\newcommand{\infinity}{\infty}
\newcommand{\ney}{\overline{NE}(Y)}
\newcommand{\nex}{\overline{NE}(X)}
\newcommand{\nes}{\overline{NE}(S)}
%%%%
\newcommand{\sub}{\subseteq}
\newcommand{\ratmap}{\dasharrow}
\newcommand{\eq}{\equiv}
\newcommand{\myquad}{\ }
%%%
%%%%%%% operators
\DeclareMathOperator{\Char}{{char}}
\DeclareMathOperator{\Cart}{{Cartier}}
\DeclareMathOperator{\fpt}{{fpt}}
\DeclareMathOperator{\lct}{{lct}}
\DeclareMathOperator{\divisor}{{div}}
\DeclareMathOperator{\Div}{{div}}
\DeclareMathOperator{\ord}{{ord}}
\DeclareMathOperator{\Frac}{{Frac}}
\DeclareMathOperator{\Ann}{{Ann}}
\DeclareMathOperator{\rd}{{rd}}
\DeclareMathOperator{\an}{{an}}
\DeclareMathOperator{\height}{{ht}}
\DeclareMathOperator{\exc}{{exc}}
\DeclareMathOperator{\coherent}{{coh}}
\DeclareMathOperator{\quasicoherent}{{qcoh}}
\DeclareMathOperator{\sn}{{sn}}
\DeclareMathOperator{\wn}{{wn}}
\DeclareMathOperator{\id}{{id}}
\DeclareMathOperator{\codim}{codim}
\DeclareMathOperator{\coker}{{coker}}
%%\DeclareMathOperator{\ker}{{ker}}
\DeclareMathOperator{\im}{{im}}
\DeclareMathOperator{\sgn}{{sgn}}
%%\DeclareMathOperator{\hom}{{Hom}}
\DeclareMathOperator{\opp}{{op}}
\DeclareMathOperator{\ext}{{Ext}}
\DeclareMathOperator{\Tor}{{Tor}}
\DeclareMathOperator{\pic}{{Pic}}
\DeclareMathOperator{\pico}{{Pic}^{\circ}}
\DeclareMathOperator{\aut}{{Aut}}
\DeclareMathOperator{\bir}{{Bir}}
\DeclareMathOperator{\lin}{{Lin}}
\DeclareMathOperator{\sym}{{Sym}}
\DeclareMathOperator{\rank}{{rank}}
\DeclareMathOperator{\rk}{{rk}}
\DeclareMathOperator{\pgl}{{PGL}}
\DeclareMathOperator{\gl}{{GL}}
\DeclareMathOperator{\Gr}{{Gr}}
\DeclareMathOperator{\ob}{{Ob}}
\DeclareMathOperator{\mor}{{Mor}}
\DeclareMathOperator{\supp}{{supp}}
\DeclareMathOperator{\Supp}{{Supp}}
\DeclareMathOperator{\Sing}{{Sing}}
\DeclareMathOperator{\var}{{Var}}
\DeclareMathOperator{\Spec}{{Spec}}
\DeclareMathOperator{\Proj}{{Proj}}
\DeclareMathOperator{\Tot}{{Tot}}
\DeclareMathOperator{\Aut}{Aut}
\DeclareMathOperator{\Lef}{Lef}
\DeclareMathOperator{\wt}{wt}
\DeclareMathOperator{\twoRC}{{RC_2^n}}
\DeclareMathOperator{\ptRC}{{RC_{\bullet}}}
\DeclareMathOperator{\twoptRC}{{RC^2_{\bullet}}}
\DeclareMathOperator{\Univ}{Univ}
\DeclareMathOperator{\Univrc}{{Univ^{rc}}}
\DeclareMathOperator{\twoUnivrc}{{Univ^{rc, 2}}}
\DeclareMathOperator{\ptUnivrc}{{Univ^{rc}_{\bullet}}}
\DeclareMathOperator{\twoptUnivrc}{{Univ_{\bullet}^{rc, 2}}}
\DeclareMathOperator{\charact}{char}
\DeclareMathOperator{\Chow}{Chow}
\DeclareMathOperator{\Dubbies}{Dubbies^n}
\DeclareMathOperator{\Ext}{Ext}
\DeclareMathOperator{\Hilb}{Hilb}
\DeclareMathOperator{\Hom}{Hom}
\DeclareMathOperator{\sHom}{{\sH}om}
\DeclareMathOperator{\Hombir}{Hom_{bir}^n}
\DeclareMathOperator{\Image}{Image}
\DeclareMathOperator{\genus}{genus}
\DeclareMathOperator{\Imaginary}{Im}
\DeclareMathOperator{\Img}{Im}
\DeclareMathOperator{\Ker}{Ker}
\DeclareMathOperator{\locus}{locus}
\DeclareMathOperator{\Num}{Num}
\DeclareMathOperator{\Pic}{Pic}
\DeclareMathOperator{\RatCurves}{RatCurves^n}
\DeclareMathOperator{\RC}{RatCurves^n}
\DeclareMathOperator{\red}{red}
\DeclareMathOperator{\Reg}{Reg}
\DeclareMathOperator{\psl}{PGL}
\DeclareMathOperator{\mult}{mult}
\DeclareMathOperator{\mld}{mld}
\renewcommand{\mod}[1]{\,(\textnormal{mod}\,#1)}
%%%%%%%%%%%%%%%%%%%%%%%%%%%%%%%%%%%%%
\def\spec#1.#2.{{\bold S\bold p\bold e\bold c}_{#1}#2}
\def\proj#1.#2.{{\bold P\bold r\bold o\bold j}_{#1}\sum #2}
\def\ring#1.{\scr O_{#1}}
\def\map#1.#2.{#1 \to #2}
\def\longmap#1.#2.{#1 \longrightarrow #2}
\def\factor#1.#2.{\left. \raise 2pt\hbox{$#1$} \right/
\hskip -2pt\raise -2pt\hbox{$#2$}}
\def\pe#1.{\mathbb P(#1)}
\def\pr#1.{\mathbb P^{#1}}
\newcommand{\sheafspec}{\mbox{\bf Spec}}
\newcommand{\len}[2]{{{\bf \ell}_{#1}{\left(#2\right)}}}

%%%%%%%%%%%%%%%%%%%%%%%%%%%%%%%%%%%%%%%%%%%%%%%%%
%%%%%% cohomology and short exact sequences %%%%%
%%%%%%%%%%%%%%%%%%%%%%%%%%%%%%%%%%%%%%%%%%%%%%%%%
\def\coh#1.#2.#3.{H^{#1}(#2,#3)}
\def\dimcoh#1.#2.#3.{h^{#1}(#2,#3)}
\def\hypcoh#1.#2.#3.{\mathbb H_{\vphantom{l}}^{#1}(#2,#3)}
\def\loccoh#1.#2.#3.#4.{H^{#1}_{#2}(#3,#4)}
\def\dimloccoh#1.#2.#3.#4.{h^{#1}_{#2}(#3,#4)}
\def\lochypcoh#1.#2.#3.#4.{\mathbb H^{#1}_{#2}(#3,#4)}
%%%%%%%%%%
\def\ses#1.#2.#3.{0  \longrightarrow  #1   \longrightarrow
 #2 \longrightarrow #3 \longrightarrow 0}
\def\sesshort#1.#2.#3.{0
 \rightarrow #1 \rightarrow #2 \rightarrow #3 \rightarrow 0}
 \def\sesa#1{0
 \rightarrow #1 \rightarrow #1 \rightarrow #1 \rightarrow 0}

%\renewcommand{\to}[1][]{\xrightarrow{\ #1\ }}
\newcommand{\onto}[1][]{\protect{\xrightarrow{\ #1\ }\hspace{-0.8em}\rightarrow}}
\newcommand{\into}[1][]{\lhook \joinrel \xrightarrow{\ #1\ }}
%%%%%%%%%%
%%%%%%%%%% iff
\def\iff#1#2#3{
    \hfil\hbox{\hsize =#1 \vtop{\noin #2} \hskip.5cm
    \lower.5\baselineskip\hbox{$\Leftrightarrow$}\hskip.5cm
    \vtop{\noin #3}}\hfil\medskip}
%%%%%%%%%%%%%%%%%%%%%%%%%%%%%%
\def\myoplus#1.#2.{\underset #1 \to {\overset #2 \to \oplus}}
\def\assign{\longmapsto}
%%%%%%%%%%%%%%%%%%%%%%%%%%%%%%
%%%%%%%%%%%%%%%%%%%%%%%%%%%%%%%%%%
%%% Arrows %%%%%%%%%%%%%%
%%%%%%%%%%%%%%%%%%%%%%%%%%%%%%%%%%%


\usepackage{verbatim}
\usepackage{enumerate}
\usepackage[normalem]{ulem}
%\usepackage{marginnote}
\newcommand{\fram}{\mathfrak{m}}
\DefineVerbatimEnvironment%
{MyVerbatim}{Verbatim}
{formatcom=\color{Violet}}


\begin{document}
\title{The  \emph{FThresholds} package for \emph{Macaulay2}}
%\author[Alberto F.\ Boix et al.]{Alberto F.\ Boix}
%\address{Department of Mathematics, Ben-Gurion University of the Negev, Beer-Sheva 8410501, Israel}
%\email{fernanal@post.bgu.ac.il}
%\thanks{A.F.\,Boix was supported by Israel Science Foundation (grant No. 844/14) and Spanish Ministerio de Econom\'ia y Competitividad MTM2016-7881-P}

\author[]{Daniel J.\ Hern\'andez}
\address{Department of Mathematics, University of Kansas, Lawrence, KS~66045, USA}
\email{hernandez@ku.edu}
\thanks{D.~J.~Hern\'andez was partially supported by NSF DMS \#1600702.}

%\author[]{Zhibek Kadyrsizova}
%\address{School of Science and Technology, Nazarbayev University, Astana, 010000, Republic of Kazakhstan}
%\email{zhibek.kadyrsizova@nu.edu.kz}
%\thanks{Z. Kadyrsizova was partially supported by NSF DMS \#1401384 and the Barbour Scholarship at the University of Michigan.}

%\author[]{Mordechai Katzman}
%\address{Department of Pure Mathematics, University of Sheffield, Sheffield S37RH, United Kingdom}
%\email{M.Katzman@sheffield.ac.uk}

%\author[]{Sara Malec}
%\address{Department of Mathematics, Hood College, Frederick, MD 21701}
%\email{malec@hood.edu}

%\author[]{Marcus Robinson}
%\address{Department of Mathematics, University of Utah, Salt Lake City, UT~84112, USA}
%\email{robinson@math.utah.edu}

\author[]{Karl Schwede}
\address{Department of Mathematics, University of Utah, Salt Lake City, UT~84112, USA}
\thanks{K.~Schwede was supported by NSF CAREER Grant DMS \#1252860/1501102, NSF FRG Grant DMS \#1265261/1501115, NSF grant \#1801849 and a Sloan Fellowship.}
\email{schwede@math.utah.edu}

%\author[]{Daniel Smolkin}
%\address{Department of Mathematics, University of Utah, Salt Lake City, UT~84112, USA}
%\email{smolkin@math.utah.edu}
%\thanks{D.~Smolkin was supported by NSF RTG Grant DMS \#1246989, NSF CAREER Grant DMS \#1252860/1501102, and NSF FRG Grant DMS \#1265261/1501115.}

\author[]{Pedro Teixeira}
\address{Department of Mathematics, Knox College, Galesburg, IL~61401, USA}
\email{pteixeir@knox.edu}

\author[]{Emily E.\ Witt}
\address{Department of Mathematics, University of Kansas, Lawrence, KS~66045, USA}
\email{witt@ku.edu}
\thanks{E.E.~Witt was partially supported by NSF DMS \#1623035.}
\date{\today}

\begin{abstract}
   This note describes the functionality implemented in the \emph{Macaulay2} package \emph{FThresholds}.
   This package is designed to compute and estimate $F$-pure thresholds, more general $F$-thresholds, and related numerical invariants arising in the study of singularities in positive characteristic commutative algebra.
\end{abstract}


\subjclass[2010]{13A35}%=

\keywords{Macaulay2, Frobenius, $F$-singularity, $F$-pure threshold, $F$-threshold}

\maketitle

\section{Introduction}

This paper describes the \emph{Macaulay2} \cite{M2} package \emph{FThresholds}, which provides tools for computing certain fundamental invariants arising in positive characteristic commutative algebra, namely, \emph{$F$-pure thresholds}, \emph{$F$-thresholds}, and \emph{$F$-jumping exponents}.
Recall that a ring $R$ of prime characteristic $p>0$ is said to be \emph{$F$-pure} if the Frobenius morphism $F \: R \to R$ sending an element to its $p^\text{th}$ power is a pure morphism (under geometric hypotheses, this is equivalent to the condition that the Frobenius morphism is split).  Regular rings are always $F$-pure, but not conversely.
The notion of $F$-purity first appeared in the work \cite{HochsterRobertsFrobeniusLocalCohomology}  of Hochster and Roberts to study local cohomology \cite{HochsterRobertsFrobeniusLocalCohomology}, was compared with rational singularities in  \cite{FedderFPureRat}, and was used globally to study properties of Schubert varieties in \cite{MehtaRamanathanFrobeniusSplittingAndCohomologyVanishing}.

After the advent of tight closure \cite{HochsterHunekeTC1}, the usage of the Frobenius endomorphism to measure singularities proliferated, and based on a strong connection discovered between $F$-pure and \emph{log canonical} singularities \cite{HaraWatanabeFRegFPure}, the notion of $F$-purity was generalized to pairs $(R, f^t)$, where $f$ is an element of $R$, and $t$ is a nonnegative real number considered a formal exponent.

Along these lines, the notion of the \emph{$F$-pure threshold} was defined in analogy with the \emph{log canonical threshold} \cite{TakagiWatanabeFPureThresh,MustataTakagiWatanabeFThresholdsAndBernsteinSato}.
The $F$-pure threshold of the pair $(R, f)$ is the supremum over all $t$ for which $(R, f^t)$ is $F$-pure.
Furthermore, \emph{$F$-threholds} were introduced, a family of invariants that naturally extends the notion of an $F$-pure threshold.

These numerical invariants are typically difficult to calculate,
and due to their significance, have been the focus of intense study over the past fifteen years.
To this end, the \emph{Macaulay2} package \emph{FThresholds} is centered around calculating and estimating the $F$-pure threshold and other $F$-thresholds, with the function {\tt fpt} at its core.


There are three main sections in this paper, which describe and illustrate the main functionality implemented in the package \emph{FThresholds}.
\begin{itemize}
\item \autoref{sec.Nu} describes the methods implemented for approximating the $F$-pure threshold and other $F$-thresholds.
\item \autoref{sec.IsFPT} explains the package's functionality for determining whether a given number is the $F$-pure threshold, or more generally, an \emph{$F$-jumping exponent}.
\item \autoref{sec.FPT} describes the features of the central function of the package, {\tt fpt}.
\end{itemize}

This package builds heavily upon the \emph{TestIdeals} package for \emph{Macaulay2}, which provides a broad range of functionality in prime characteristic commutative algebra.

\subsection*{Acknowledgements.}  First and foremost, the authors enthusiastically thank all authors and contributors to the \emph{FThresholds} package: (INSERT NAMES HERE). %Erin Bela, Juliette Bruce, Zhibek Kadyrsizova, Sara Malec, Maral Mostafazadehfard, Marcus Robinson, Dan Smolkin, and Robert Walker.
We are also grateful the organizers of the \emph{Macaulay2} workshops where much of the functionality described herein was developed, hosted by Wake Forest University in 2012, the University of California, Berkeley in 2014 and 2017, Boise State University in 2015, and the University of Utah in 2016.
We are also grateful to the University of Utah for hosting a collaborative development visit in 2018, and to the Institute of Mathematics and its Applications for its support for the Coding Sprint \emph{F-thresholds in Macaulay2} in 2019, where the current version of the package was finalized and this paper was completed.

\section{The $F$-pure threshold limit and {\tt nu}}
\label{sec.Nu}

Suppose $R = k[x_1, \dots, x_n]$ is a polynomial ring over a finite field and $f \in R$.    If $f^a \notin \langle x_1^{p^e}, \dots, x_n^{p^e} \rangle = \fram^{[p^e]}$, then the pair $(R, f^{a/(p^e -1)})$ is $F$-pure at the origin, whereas if $f^a \in \fram^{[p^e]}$, then the pair $(R, f^{a/p^e})$ is not $F$-pure at the origin.  By finding the $a \in \bZ$ where that transition happens, and then limiting over $e$, one can compute the $F$-pure threshold.  See \cite{MustataTakagiWatanabeFThresholdsAndBernsteinSato}.

We set $\nu(e, f)$ to be largest $a \in \bZ$ such that $f^a \notin \fram^{[p^e]}$ (note in the literature, it is frequently denoted by $\nu_f(p^e)$.  It immediately follows that
\[
\fpt(f) = \lim_{e \to \infty} {\nu(e, f) \over p^e}.
\]
These numbers $\nu(e, f)$ can be computed using the command {\tt nu}, as in the example below.
\medskip
{\small
\setstretch{.67}
\begin{MyVerbatim}
i1 : loadPackage "FThresholds"

o1 = FThresholds

o1 : Package

i2 : R = ZZ/7[x,y]

o2 = R

o2 : PolynomialRing

i3 : f = y^2 - x^3

        3    2
o3 = - x  + y

o3 : R

i4 : nu(1, f)

o4 = 5

o4 : QQ

i5 : nu(2, f)

o5 = 40

o5 : QQ

i6 : nu(3, f)

o6 = 285

o6 : QQ
\end{MyVerbatim}
}
\medskip
The numbers $5/7, 40/49, 285/343$ then approximate the $F$-pure threshold, which in this case happens to equal $5/6$.  By default, if the function $f$ has a special form (is diagonal or binomial), then the function does not check whether $f^a$ is in  $\langle x_1^{p^e}, \dots, x_n^{p^e} \rangle$, but instead uses general formulas for $\nu$s, see \cite{HernandezFPureThresholdOfBinomial,HernandezFInvariantsOfDiagonalHyp}.

In fact, one can make the following more general definition.  If $I$ and $J$ are two ideals (with $I \subseteq \sqrt{J}$), one can define
\[
\nu(e, I, J) = \max \{ n \in \bZ \; |\; I^n \not\subseteq J^{[p^e]} \}.
\]
This is frequently denoted by $\nu_I^J(p^e)$ in the literature.  In the case that $I = \langle f \rangle$ and $J = \fram$, this just recovers the $\nu$'s defined above; the limit of which is the $F$-pure threshold.  For more general $I$ and $J$, the limit
\[
\lim_{e \to \infty} {\nu(e, I, J) \over p^e}
\]
is called the \emph{$F$-threshold of $I$ with respect to $J$}.  To compute these values we use a binary search, unless the user specifies that we should use a linear search instead {\tt Search=>Linear}.

There is one more option that deserves mentioning, {\tt ContainmentTest}.  There are two ways to check whether $f^a$ (or $I^a$) is contained in $J^{[p^e]}$.  In general it is true that
\[
I^a \subseteq J^{[p^e]} \text{ $\Leftrightarrow$ } (I^a)^{[1/p^e]} \subseteq J
\]
where $\bullet^{[1/p^e]}$ is defined as in the {\tt TestIdeals} package.  Which of these two strategies to use is controlled by the option {\tt ContainmentTest => StandardPower} and {\tt ContainmentTest => FrobeniusRoot} respectively.
For principal $I$, the command on the right is generally faster (and so {\tt FrobeniusRoot} is the default strategy for computing $\nu$).  For non-principal $I$, frequently the command on the left is quicker, and so {\tt StandardPower} is default strategy.

Finally, the option {\tt ContainmentTest => FrobeniusPower}, then ${\tt \nu(e, I, A)}$ computes the largest integer $n$ such that
\[
I^{[p^n]} \subseteq J^{[p^e]}.
\]
See \cite{HernandezTeixeiraWittFrobeniusPowers} for more discussion of this.

\section{{\tt isFPT}, {\tt compareFPT} and {\tt isFJumpingExponent}}
\label{sec.IsFPT}

The command {\tt isFPT(t, f)} checks whether $t$ is the $F$-pure threshold of $f$.  In general this does this by computing two test ideals:
\[
\tau(R, f^t) \text{ and } \tau(R, f^{t-\epsilon}).
\]
It follows that $\fpt(f) = t$ if and only if $\tau(R, f^t) \neq R$ and $\tau(R, f^{t-\epsilon}) = R$.  The function computes these two values by using the {\tt testIdeal} and {\tt FPureModule} commands from the {\tt TestIdeals} package.  Note that in a strongly $F$-regular Gorenstein ambient ring, {\tt FPureModule(t, f)} computes exactly $\tau(R, f^{t-\epsilon})$.  These commands work more generally, but they \emph{only} currently work when $R$ is $\bQ$-Gorenstein with index not divisible by the characteristic $p > 0$.  A more general version of this command is {\tt compareFPT(t, f)} which returns $-1$ if $t < \fpt(f)$, $0$ if $t = \fpt(f)$ and $1$ if $t > \fpt(f)$.

Consider the following example which has $F$-pure threshold equal to {1/2}, see \cite[Example 4.3]{MustataYoshidaTestIdealVsMultiplierIdeals}.
\medskip
{\small
\setstretch{.67}
\begin{MyVerbatim}
i1 : loadPackage "FThresholds"

o1 = FThresholds

o1 : Package

i2 : R = ZZ/2[x,y,z];

i3 : f = x^2+y^5+z^5;

i4 : isFPT(1/2, f)

o4 = true

i5 : isFPT(1/3, f)

o5 = false

i6 : compareFPT(3/4, f)

o6 = 1
\end{MyVerbatim}
}
\medskip

The command {\tt isFJumpingExponent} is implemented similarly, a value $t$ is an $F$-jumping exponent if and only if
\[
\tau(R, f^t) \neq \tau(R, f^{t-\epsilon}).
\]
Again, this command works as long as $R$ is $\bQ$-Gorenstein with index not divisible by $p$.


\newpage
\section{The {\tt fpt} function}
\label{sec.FPT}

The core function in this package is the {\tt fpt} function.  Throughout this section, let $f$ be a polynomial with coefficients in a finite field of characteristic $p$. When passed the polynomial $f$, the function {\tt fpt} attempts to find the exact value for the $F$-pure threshold of $f$ at the origin, and returns that value, if possible.  Otherwise, it returns lower and upper bounds for the $F$-pure threshold, as demonstrated below.

{\small
\setstretch{.67}
\begin{MyVerbatim}

i1 : ZZ/5[x,y,z];

i2 : fpt( x^3 + y^3 + z^3 + x*y*z )

     4
o2 = -
     5

o2 : QQ

i3 : fpt( x^5 + y^6 + z^7 + (x*y*z)^3 )

      1  2
o3 = {-, -}
      3  5

o3 : List

\end{MyVerbatim}
}

             If the option {\tt UseSpecialAlgorithms} is set to {\tt true} (the default value), then {\tt fpt} first checks whether $f$ is diagonal polynomial, a binomial, or a form in two variables, in that order.
             If it is one of these, algorithms of Hern\'andez \cite{HernandezFInvariantsOfDiagonalHyp, HernandezFPureThresholdOfBinomial}, or Hern\'andez and Teixeira \cite{HernandezTeixeiraFThresholdFunctions},  are executed to compute the $F$-pure threshold (cf. \cite{ShibutaTakagiLCThresholds}).

{\small
\setstretch{.67}
\begin{MyVerbatim}

i4 : fpt( x^17 + y^20 + z^24 ) -- a diagonal polynomial

      94
o4 = ---
     625

i5 : fpt( x^2*y^6*z^10 + x^10*y^5*z^3 ) -- a binomial

      997
o5 = ----
     6250

o5 : QQ

i6 : ZZ/5[x,y];

i7 : fpt( x^2*y^6*(x + y)^9*(x + 3*y)^10 ) -- a form in two variables

      5787
o7 = -----
     78125

o7 : QQ

\end{MyVerbatim}
}

            The above noted algorithm for computing the $F$-pure threshold of a binary form $f$ requires factoring $f$ into linear forms, and {\tt fpt} can sometimes hang when attempting that factorization. For this reason, when a factorization is already known, the user can pass to {\tt fpt} a list containing all the pairwise prime linear factors of $f$, and a list containing their respective multiplicities.

{\small
\setstretch{.67}
\begin{MyVerbatim}

i8 : L = {x, y, x + y, x + 3*y};

i9 : m = {2, 6, 9, 10};

i10 : fpt(L, m)==o7

o10 = true

\end{MyVerbatim}
}

In the remainder of this section, we describe the {\tt fpt} function when no special algorithm is available, or {\tt UseSpecialAlgorithms} is set to {\tt false},  as well as the roles of the options {\tt DepthOfSearch}, and {\tt Attempts}, both nonnegative integers.

In this case, very roughly speaking, the {\tt fpt} function will either compute, and output, the exact value of the $F$-pure threshold of $f$, and otherwise recursively compute a finite sequence of lower and upper bounds for this value, and output the last of these, which will be the tightest among all computed.  The value of the option {\tt DepthOfSearch} determines the precision of the initial set of bounds, and the option {\tt Attempts}, in conjunction with some various subroutines, determines whether, and how, to produce new, tighter bounds from the previous ones.

In more detail, let $e$ denote the value of the option {\tt DepthOfSearch}, which conservatively defaults to {\tt 1}.  The {\tt fpt} function first computes $\nu=\nu_f(p^e)$, which agrees with the output of {\tt nu(e,f)}.  It is well known that the $F$-pure threshold of $f$ is greater than $\nu/p^e$ and at most $(\nu+1)/p^e$, and applying  \cite[Proposition 4.2]{HernandezFPurityOfHypersurfaces} to this lower bound tells us that the $F$-pure threshold of $f$ must be at least $\nu/(p^e-1)$.  In summary, we know that the $F$-pure threshold of $f$ must lie in the closed interval
%
\begin{equation}
\label{estimating-interval: e}
\tag{$\dagger$}
\left[ \frac{\nu}{p^e-1}, \frac{\nu+1}{p^e} \right].
\end{equation}

With these estimates in hand, the subroutine {\tt guessFPT} is called to make some ``educated guesses" in an attempt to identify the $F$-pure threshold within this interval, or at least narrow down this interval to produce improved estimates.  The number of ``guesses" is controlled by the option {\tt Attempts}, which conservatively defaults to {\tt 3}.  If {\tt Attempts} is set to {\tt 0}, then {\tt guessFPT} is bypassed. If  {\tt Attempts} is set to at least {\tt 1}, then a first check is run to verify whether the right-hand endpoint $(\nu+1)/p^e$ of the above interval \eqref{estimating-interval: e} is the $F$-pure threshold.  We illustrate this below.

{\small
\setstretch{.67}
\begin{MyVerbatim}

i11 : f = x^2*(x + y)^3*(x + 3*y^2)^5;

i12 : fpt( f, Attempts => 0 ) -- a bad estimate

          1
o12 = {0, -}
          5

o12 : List

i13 : fpt( f, Attempts => 0, DepthOfSearch => 3 ) -- a better estimate

        21   22
o13 = {---, ---}
       124  125

o13 : List

i14 : fpt( f, Attempts => 1, DepthOfSearch => 3 ) -- the right-hand

      endpoint (nu+1)/p^e is the F-pure threshold

       22
o14 = ---
      125

o14 : QQ

\end{MyVerbatim}
}

If  {\tt Attempts} is set to at least {\tt 2} and the right-hand endpoint $(\nu+1)/p^e$ of the interval \eqref{estimating-interval: e} is not the $F$-pure threshold, then a second check is run to verify whether the left-hand endpoint $\nu/(p^e-1)$ of this interval is the $F$-pure threshold.

{\small
\setstretch{.67}
\begin{MyVerbatim}

i15 : f = x^6*y^4 + x^4*y^9 + (x^2 + y^3)^3;

i16 : fpt( f, Attempts => 1, DepthOfSearch => 3 )

       17   7
o16 = {--, --}
       62  25

o16 : List

i17 : fpt( f, Attempts => 2, DepthOfSearch => 3 ) -- the left-hand endpoint

      nu/(p^e-1) is the F-pure threshold

      17
o17 = --
      62

o17 : QQ

\end{MyVerbatim}
}

            If neither endpoint is the $F$-pure threshold and {\tt Attempts} is set to more than {\tt 2}, then  additional checks are performed at certain numbers within the interval .  A number in the interval \eqref{estimating-interval: e} with minimal denominator is selected, and {\tt compareFPT} is used to test that number. {\color{red} This will need to get updated} If that ``guess" is correct, its value is returned; otherwise, the information returned by {\tt compareFPT} is used to narrow down the interval, and this process is repeated as many times as specified by {\tt Attempts}.

{\small
\setstretch{.67}
\begin{MyVerbatim}

i18 : f = x^3*y^11*(x + y)^8*(x^2 + y^3)^8;

i19 : fpt( f, DepthOfSearch => 3, Attempts => 2 )

        3   7
o19 = {--, ---}
       62  125

o19 : List

i20 : fpt( f, DepthOfSearch => 3, Attempts => 3 ) -- an additional check

      sharpens the estimate

        3   1
o20 = {--, --}
       62  18

o20 : List

i21 : fpt( f, DepthOfSearch => 3, Attempts => 4 ) -- and one more finds

      the answer

       1
o21 = --
      19

o21 : QQ

\end{MyVerbatim}
}


If {\tt guessFPT} is unsuccessful and {\tt UseFSignature} is set to {\tt true}, then the {\tt fpt} function proceeds to use the convexity of the $F$-signature function and a secant line argument to attempt to narrow down the interval bounding the $F$-pure threshold.

{\small
\setstretch{.67}
\begin{MyVerbatim}

i22 : f = x^5*y^6*(x + y)^9*(x^2 + y^3)^4;

i23 : fpt( f, DepthOfSearch => 3 )

        2   1
o23 = {--, --}
       31  14

o23 : List

i24 : fpt( f, DepthOfSearch => 3, UseFSignature => true )

        181   1
o24 = {----, --}
       2750  14

o24 : List

i25 : numeric o23

o25 = {.0645161, .0714286}

o25 : List

i26 : numeric o24 -- UseFSignature sharpened the estimate a bit

o26 = {.0658182, .0714286}

o26 : List

\end{MyVerbatim}
}

When {\tt FRegularityCheck} is set to {\tt true} and no exact answer has been found, a final check is run to verify whether the final lower bound for the $F$-pure threshold is the exact answer (if it has not already been eliminated as a possibility).

{\small
\setstretch{.67}
\begin{MyVerbatim}

i27 : f = (x + y)^4*(x^2 + y^3)^6;

i28 : fpt( f, Attempts => 2, DepthOfSearch => 3 )

        3   13
o28 = {--, ---}
       31  125

o28 : List

i29 : fpt( f, Attempts => 2, DepthOfSearch => 3, UseFSignature => true ) --

      UseFSignature improves the answer a bit

        1   13
o29 = {--, ---}
       10  125

o29 : List

i30 : fpt( f, Attempts => 2, DepthOfSearch => 3, UseFSignature => true,

      FRegularityCheck => true ) -- FRegularityCheck finds the answer

       1
o30 = --
      10

o30 : QQ

\end{MyVerbatim}
}

The computations performed when {\tt UseFSignature} and {\tt FRegularityCheck} are set to {\tt true} are often slow, and often fail to improve the estimate, and for this reason, these options should be used sparingly.
            It is often more effective to increase the values of {\tt Attempts} or {\tt DepthOfSearch} instead.

{\small
\setstretch{.67}
\begin{MyVerbatim}

i31 : f = x^7*y^5*(x + y)^5*(x^2 + y^3)^4;

i32 : timing numeric fpt( f, DepthOfSearch => 3, UseFSignature => true,

      FRegularityCheck => true )

o32 = {.0733061, .0769231}
      -- 2.68899 seconds

o32 : Time

i33 : timing numeric fpt( f, Attempts => 5, DepthOfSearch => 3 ) -- a better

      answer in less time

o33 = {.075, .0769231}
      -- .893389 seconds

o33 : Time

i34 : timing fpt( f, DepthOfSearch => 4 ) -- the exact answer in even less
      time

       48
o34 = ---
      625
      -- .361882 seconds

o34 : Time

\end{MyVerbatim}
}

As seen in several examples above, when the exact  $F$-pure threshold of $f$ is not found, a list containing the endpoints of an interval containing its value is returned.  Whether that interval is open, closed, or a mixed interval depends on the options passed; if the option {\tt Verbose} is set to {\tt true}, the precise interval will be printed.

{\small
\setstretch{.67}
\begin{MyVerbatim}

i35 : f = x^7*y^5*(x + y)^5*(x^2 + y^3)^4;

i36 : fpt( f, DepthOfSearch => 3, UseFSignature => true, Verbose => true )

Starting fpt ...

fpt is not 1 ...

Verifying if special algorithms apply...

Special fpt algorithms were not used ...

nu has been computed: nu = nu(3,f) = 9 ...

fpt lies in the interval [ nu/(p^e-1), (nu+1)/p^e ] = [ 9/124, 2/25 ] ...

Starting guessFPT ...

The right-hand endpoint is not the fpt ...

The left-hand endpoint is not the fpt ...

guessFPT narrowed the interval down to ( 9/124, 1/13 ) ...

Beginning F-signature computation ...

First F-signature computed: s(f,(nu-1)/p^e) = 456/15625 ...

Second F-signature computed: s(f,nu/p^e) = 64/15625 ...

Computed F-signature secant line intercept: 449/6125 ...

F-signature intercept is an improved lower bound ...

fpt failed to find the exact answer; try increasing the value of

DepthOfSearch or Attempts.

fpt lies in the interval [ 449/6125, 1/13 ).

        449   1
o36 = {----, --}
       6125  13

o36 : List

\end{MyVerbatim}
}

\newpage
\bibliographystyle{skalpha}
\bibliography{MainBib}



\end{document}
