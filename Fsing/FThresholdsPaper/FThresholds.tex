\documentclass{amsart}
\usepackage{calc,amssymb,amsthm,amsmath,setspace}
%\usepackage{mathtools}
\RequirePackage[dvipsnames,usenames]{xcolor}

\usepackage{fancyvrb}
\usepackage{hyperref}
\hypersetup{
bookmarks,
bookmarksdepth=3,
bookmarksopen,
bookmarksnumbered,
pdfstartview=FitH,
colorlinks,backref,hyperindex,
linkcolor=Sepia,
anchorcolor=BurntOrange,
citecolor=MidnightBlue,
citecolor=OliveGreen,
filecolor=BlueViolet,
menucolor=Yellow,
urlcolor=OliveGreen
}
\usepackage{xspace}
%\usepackage{rotating}
\interfootnotelinepenalty=100000

\usepackage{mabliautoref}
\usepackage{colonequals}
\frenchspacing
\input{kmacros3.sty}
\usepackage{stmaryrd}

\usepackage{verbatim}
\usepackage{enumerate}

\usepackage[normalem]{ulem}
%\usepackage{marginnote}

\DeclareMathOperator{\HH}{H}
\newcommand{\fram}{\mathfrak{m}}
%\DeclareMathOperator{\Image}{image}

\DefineVerbatimEnvironment%
  {MyVerbatim}{Verbatim}
  {formatcom=\color{Violet}}

\renewcommand{\geq}{\geqslant}
\renewcommand{\leq}{\leqslant}
\renewcommand{\ge}{\geqslant}
\renewcommand{\le}{\leqslant}

\newtheorem*{caveat}{Caveat}

\begin{document}
\title{The  \emph{TestIdeals} package for \emph{Macaulay2}}
\author[Alberto F.\ Boix et al.]{Alberto F.\ Boix}
\address{Department of Mathematics, Ben-Gurion University of the Negev, Beer-Sheva 8410501, Israel}
\email{fernanal@post.bgu.ac.il}
\thanks{A.F.\,Boix was supported by Israel Science Foundation (grant No. 844/14) and Spanish Ministerio de Econom\'ia y Competitividad MTM2016-7881-P}

\author[]{Daniel J.\ Hern\'andez}
\address{Department of Mathematics, University of Kansas, Lawrence, KS~66045, USA}
\email{hernandez@ku.edu}
\thanks{D.~J.~Hern\'andez was partially supported by NSF DMS \#1600702.}

\author[]{Zhibek Kadyrsizova}
\address{School of Science and Technology, Nazarbayev University, Astana, 010000, Republic of Kazakhstan}
\email{zhibek.kadyrsizova@nu.edu.kz}
\thanks{Z. Kadyrsizova was partially supported by NSF DMS \#1401384 and the Barbour Scholarship at the University of Michigan.}

\author[]{Mordechai Katzman}
\address{Department of Pure Mathematics, University of Sheffield, Sheffield S37RH, United Kingdom}
\email{M.Katzman@sheffield.ac.uk}

\author[]{Sara Malec}
\address{Department of Mathematics, Hood College, Frederick, MD 21701}
\email{malec@hood.edu}

\author[]{Marcus Robinson}
\address{Department of Mathematics, University of Utah, Salt Lake City, UT~84112, USA}
\email{robinson@math.utah.edu}

\author[]{Karl Schwede}
\address{Department of Mathematics, University of Utah, Salt Lake City, UT~84112, USA}
\thanks{K.~Schwede was supported by NSF CAREER Grant DMS \#1252860/1501102, NSF FRG Grant DMS \#1265261/1501115, NSF grant \#1801849 and a Sloan Fellowship.}
\email{schwede@math.utah.edu}

\author[]{Daniel Smolkin}
\address{Department of Mathematics, University of Utah, Salt Lake City, UT~84112, USA}
\email{smolkin@math.utah.edu}
\thanks{D.~Smolkin was supported by NSF RTG Grant DMS \#1246989, NSF CAREER Grant DMS \#1252860/1501102, and NSF FRG Grant DMS \#1265261/1501115.}

\author[]{Pedro Teixeira}
\address{Department of Mathematics, Knox College, Galesburg, IL~61401, USA}
\email{pteixeir@knox.edu}

\author[]{Emily E.\ Witt}
\address{Department of Mathematics, University of Kansas, Lawrence, KS~66045, USA}
\email{witt@ku.edu}
\thanks{E.E.~Witt was partially supported by NSF DMS \#1623035.}
\date{\today}

\begin{abstract}
	This note describes a \emph{Macaulay2} package for computations of $F$(-pure) thresholds in characteristic $p > 0$ commutative algebra.  
\end{abstract}


\subjclass[2010]{13A35}%=

\keywords{Macaulay2}

\maketitle

\section{Introduction}

This paper describes a Macaulay2 package which provides tools to compute, $F$-pure thresholds, $F$-thresholds and $F$-jumping exponents.  Recall first that a ring $R$ is $F$-pure if the Frobenius morphism $F : R \to R$ is a pure morphism (under geometric hypotheses, this is equivalent to Frobenius being split).  The notion of $F$-purity appeared first in the work of Hochster and Roberts in \cite{HochsterRobertsFrobeniusLocalCohomology}, was compared with rational singularities in \cite{FedderFPureRat} and was used globally to study properties of Schubert varieties in \cite{MehtaRamanathanFrobeniusSplittingAndCohomologyVanishing}.  After the advent of tight closure \cite{HochsterHunekeTC1}, the usage of Frobenius to measure singularities proliferated and based on a tight connection discovered between $F$-pure and log canonical singularities \cite{HaraWatanabeFRegFPure}, the notion of $F$-purity was generalized to pairs $(R, f^t)$ where $f \in R$ and $t \in \bQ_{\geq 0}$ is a formal exponent.  Along these lines the notion of the \emph{$F$-pure threshold} (or $\fpt$) was defined in analogy with the log canonical threshold (or $\lct$) \cite{TakagiWatanabeFPureThresh,MustataTakagiWatanabeFThresholdsAndBernsteinSato}.  The $\fpt$ of $(R, f)$ is the supremum over all $t$ such that $(R, f^t)$ is $F$-pure, and the functionality computing this number is the core of this package.    As a further generalization, the notion of \emph{$F$-threshold} was also introduced.  All of these \emph{thresholds} have been the focus of intense study over the past 15 years.

There are three main sections in this paper
\begin{itemize}
\item{} \autoref{sec.Nu} which describes how to first approximate the $\fpt$ and other $F$-thresholds.  
\item{} \autoref{sec.IsFPT} describes functionality which can determine whether a given number is the FPT (or more generally, an $F$-jumping exponent).
\item{} \autoref{sec.FPT} describes the {\tt fpt} function and how it works.
\end{itemize}

This package builds heavily upon the {\tt TestIdeals} package.  

{\it Acknowledgements}  The authors thank...  Thank the IMA...

\section{The $F$-pure threshold limit and {\tt nu}}
\label{sec.Nu}

Suppose $R = k[x_1, \dots, x_n]$ is a polynomial ring over a finite field and $f \in R$.    If $f^a \notin \langle x_1^{p^e}, \dots, x_n^{p^e} \rangle = \fram^{[p^e]}$ then the pair $(R, f^{a/(p^e -1)})$ is $F$-pure at the origin, whereas if $f^a \in \fram^{[p^e]}$, then the pair $(R, f^{a/p^e})$ is not $F$-pure at the origin.  By finding the $a \in \bZ$ where that transition happens, and then limiting over $e$, one can compute the $F$-pure threshold.  See \cite{MustataTakagiWatanabeFThresholdsAndBernsteinSato}.

We set $\nu(e, f)$ to be largest $a \in \bZ$ such that $f^a \notin \fram^{[p^e]}$ (note in the literature, it is frequently denoted by $\nu_f(p^e)$.  It immediately follows that 
\[
\fpt(f) = \lim_{e \to \infty} {\nu(e, f) \over p^e}.
\]
These numbers $\nu(e, f)$ can be computed using the command {\tt nu}, as in the example below.  
\medskip
{\small
\setstretch{.67}
\begin{MyVerbatim}
i1 : loadPackage "FThresholds"

o1 = FThresholds

o1 : Package

i2 : R = ZZ/7[x,y]

o2 = R

o2 : PolynomialRing

i3 : f = y^2 - x^3

        3    2
o3 = - x  + y

o3 : R

i4 : nu(1, f)

o4 = 5

o4 : QQ

i5 : nu(2, f)

o5 = 40

o5 : QQ

i6 : nu(3, f)

o6 = 285

o6 : QQ
\end{MyVerbatim}
}
\medskip
The numbers $5/7, 40/49, 285/343$ then approximate the $F$-pure threshold, which in this case happens to equal $5/6$.  By default, if the function $f$ has a special form (is diagonal or binomial), then the function does not check whether $f^a$ is in  $\langle x_1^{p^e}, \dots, x_n^{p^e} \rangle$, but instead uses general formulas for $\nu$s, see \cite{HernandezFPureThresholdOfBinomial,HernandezFInvariantsOfDiagonalHyp}.

In fact, one can make the following more general definition.  If $I$ and $J$ are two ideals (with $I \subseteq \sqrt{J}$), one can define
\[
\nu(e, I, J) = \max \{ n \in \bZ \; |\; I^n \not\subseteq J^{[p^e]} \}.
\]
This is frequently denoted by $\nu_I^J(p^e)$ in the literature.  In the case that $I = \langle f \rangle$ and $J = \fram$, this just recovers the $\nu$'s defined above; the limit of which is the $F$-pure threshold.  For more general $I$ and $J$, the limit
\[
\lim_{e \to \infty} {\nu(e, I, J) \over p^e}
\]
is called the \emph{$F$-threshold of $I$ with respect to $J$}.  To compute these values we use a binary search, unless the user specifies that we should use a linear search instead {\tt Search=>Linear}.  

There is one more option that deserves mentioning, {\tt ContainmentTest}.  There are two ways to check whether $f^a$ (or $I^a$) is contained in $J^{[p^e]}$.  In general it is true that
\[
I^a \subseteq J^{[p^e]} \text{ $\Leftrightarrow$ } (I^a)^{[1/p^e]} \subseteq J
\]
where $\bullet^{[1/p^e]}$ is defined as in the {\tt TestIdeals} package.  Which of these two strategies to use is controlled by the option {\tt ContainmentTest => StandardPower} and {\tt ContainmentTest => FrobeniusRoot} respectively.
For principal $I$, the command on the right is generally faster (and so {\tt FrobeniusRoot} is the default strategy for computing $\nu$).  For non-principal $I$, frequently the command on the left is quicker, and so {\tt StandardPower} is default strategy. 

Finally, the option {\tt ContainmentTest => FrobeniusPower}, then ${\tt \nu(e, I, A)}$ computes the largest integer $n$ such that
\[
I^{[p^n]} \subseteq J^{[p^e]}.
\]
See \cite{HernandezTeixeiraWittFrobeniusPowers} for more discussion of this.  

\section{{\tt isFPT}, {\tt compareFPT} and {\tt isFJumpingExponent}}
\label{sec.IsFPT}

The command {\tt isFPT(t, f)} checks whether $t$ is the $F$-pure threshold of $f$.  In general this does this by computing two test ideals:
\[
\tau(R, f^t) \text{ and } \tau(R, f^{t-\epsilon}).
\]
It follows that $\fpt(f) = t$ if and only if $\tau(R, f^t) \neq R$ and $\tau(R, f^{t-\epsilon}) = R$.  The function computes these two values by using the {\tt testIdeal} and {\tt FPureModule} commands from the {\tt TestIdeals} package.  Note that in a strongly $F$-regular Gorenstein ambient ring, {\tt FPureModule(t, f)} computes exactly $\tau(R, f^{t-\epsilon})$.  These commands work more generally, but they \emph{only} currently work when $R$ is $\bQ$-Gorenstein with index not divisible by the characteristic $p > 0$.  A more general version of this command is {\tt compareFPT(t, f)} which returns $-1$ if $t < \fpt(f)$, $0$ if $t = \fpt(f)$ and $1$ if $t > \fpt(f)$.  

Consider the following example which has $F$-pure threshold equal to {1/2}, see \cite[Example 4.3]{MustataYoshidaTestIdealVsMultiplierIdeals}.
\medskip
{\small
\setstretch{.67}
\begin{MyVerbatim}
i1 : loadPackage "FThresholds"

o1 = FThresholds

o1 : Package

i2 : R = ZZ/2[x,y,z];

i3 : f = x^2+y^5+z^5;

i4 : isFPT(1/2, f)

o4 = true

i5 : isFPT(1/3, f)

o5 = false

i6 : compareFPT(3/4, f)

o6 = 1
\end{MyVerbatim}
}
\medskip

The command {\tt isFJumpingExponent} is implemented similarly, a value $t$ is an $F$-jumping exponent if and only if
\[
\tau(R, f^t) \neq \tau(R, f^{t-\epsilon}).  
\]
Again, this command works as long as $R$ is $\bQ$-Gorenstein with index not divisible by $p$. 


\section{\tt fpt}
\label{sec.FPT}

The core function in this package is the {\tt fpt} function.  Given an element $f \in R = k[x_1, \dots, x_n]$, the {\tt fpt} function tries to compute the $F$-pure threshold of $f$.  It does this in the following way:

\section{}


\bibliographystyle{skalpha}
\bibliography{MainBib}



\end{document}
