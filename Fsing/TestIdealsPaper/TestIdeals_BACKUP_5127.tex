\documentclass[11pt]{amsart}
\usepackage{calc,amssymb,amsthm,amsmath,fullpage    }
%\usepackage{mathtools}
\RequirePackage[dvipsnames,usenames]{xcolor}
\usepackage{hyperref}
\hypersetup{
bookmarks,
bookmarksdepth=3,
bookmarksopen,
bookmarksnumbered,
pdfstartview=FitH,
colorlinks,backref,hyperindex,
linkcolor=Sepia,
anchorcolor=BurntOrange,
citecolor=MidnightBlue,
citecolor=OliveGreen,
filecolor=BlueViolet,
menucolor=Yellow,
urlcolor=OliveGreen
}
\usepackage{alltt}
\usepackage{multicol}
%\usepackage{etex}
\usepackage{xspace}
\usepackage{rotating}
\interfootnotelinepenalty=100000

\usepackage{mabliautoref}
\usepackage{colonequals}
\frenchspacing
%%   Setup to use Ralph Smith Formal Script font:
%%   Provided by Sandor Kovacs to Karl Schwede, modified further
\DeclareFontFamily{OMS}{rsfs}{\skewchar\font'60}
\DeclareFontShape{OMS}{rsfs}{m}{n}{<-5>rsfs5 <5-7>rsfs7 <7->rsfs10 }{}
\DeclareSymbolFont{rsfs}{OMS}{rsfs}{m}{n}
\DeclareSymbolFontAlphabet{\scr}{rsfs}

% \newenvironment{pf}[1][{\it Proof:}]{\begin{trivlist}
% \item[\hskip \labelsep {\bfseries #1}]}{\end{trivlist}}

%%***Collaborative editing commands***
\newcommand{\note}[1]{\marginpar{\sffamily\tiny #1}}

\def\mytodo#1{\textcolor{Mahogany}%
{\sffamily\footnotesize\newline{\color{Mahogany}\fbox{\parbox{\textwidth-15pt}{#1}}}\newline}}


%%***Common Names***
\newcommand{\Cech}{{$\check{\text{C}}$ech} }
\newcommand{\mustata}{Musta{\c{t}}{\u{a}}}
\newcommand{\etale}{{\'e}tale }


%%%************Functors, derived categories and related****************
\newcommand{\myR}{{\bf R}}
\newcommand{\myH}{{\bf h}}
\newcommand{\myL}{{\bf L}}
\newcommand{\qis}{\simeq_{\text{qis}}}
\newcommand{\tensor}{\otimes}
\newcommand{\hypertensor}{{\uuline \tensor}}
\newcommand{\mydot}{{{\,\begin{picture}(1,1)(-1,-2)\circle*{2}\end{picture}\ }}}
\newcommand{\blank}{\underline{\hskip 10pt}}
\DeclareMathOperator{\trunc}{{trunc}}

%%*************Diagram of scheme notation
\DeclareMathOperator{\kSch}{{Sch}}
\DeclareMathOperator{\kCat}{{Cat}}
\DeclareMathOperator{\kHrc}{{Hrc}}
\newcommand{\uSch}{\underline{\kSch}}
\newcommand{\uCat}{\underline{\kCat}}
\newcommand{\uHrc}{\underline{\kHrc}}
\newcommand{\Ho}{\text{Ho}}

%%%*************Du Bois short hand******************
\newcommand{\DuBois}[1]{{\uuline \Omega {}^0_{#1}}}
\newcommand{\FullDuBois}[1]{{\uuline \Omega {}^{\mydot}_{#1}}}

\newcommand{\modGens}[2]{{{\bf \mu}_{#1}{\left(#2\right)}}}


%%%*************F-singularity short hand************
\newcommand{\tauCohomology}{T}
\newcommand{\FCohomology}{S}
\newcommand{\Ram}{\mathrm{Ram}}
\DeclareMathOperator{\Tr}{Tr}
\newcommand{\roundup}[1]{\lceil #1 \rceil}
\newcommand{\rounddown}[1]{\lfloor #1 \rfloor}
\newcommand{\HK}{\mathrm{HK}}

%%***************Latin short hand*******************
\newcommand{\cf}{{\itshape cf.} }
\newcommand{\loccit}{{\itshape loc. cit.} }
\newcommand{\ie}{{\itshape i.e.} }

%%************Script**********************
\newcommand{\sA}{\scr{A}}
\newcommand{\sB}{\scr{B}}
\newcommand{\sC}{\scr{C}}
\newcommand{\sD}{\scr{D}}
\newcommand{\sE}{\scr{E}}
\newcommand{\sF}{\scr{F}}
\newcommand{\sG}{\scr{G}}
\newcommand{\sH}{\scr{H}}
\newcommand{\sI}{\scr{I}}
\newcommand{\sJ}{\scr{J}}
\newcommand{\sK}{\scr{K}}
\newcommand{\sL}{\scr{L}}
\newcommand{\sM}{\scr{M}}
\newcommand{\sN}{\scr{N}}
\newcommand{\sO}{\scr{O}}
\newcommand{\sP}{\scr{P}}
\newcommand{\sQ}{\scr{Q}}
\newcommand{\sR}{\scr{R}}
\newcommand{\sS}{\scr{S}}
\newcommand{\sT}{\scr{T}}
\newcommand{\sU}{\scr{U}}
\newcommand{\sV}{\scr{V}}
\newcommand{\sW}{\scr{W}}
\newcommand{\sX}{\scr{X}}
\newcommand{\sY}{\scr{Y}}
\newcommand{\sZ}{\scr{Z}}

\newcommand{\enm}[1]{\ensuremath{#1}}
\newcommand{\mJ}{\mathcal{J}}
\newcommand{\uTwo}{\underline{2}}
\newcommand{\uOne}{\underline{1}}
\newcommand{\uZero}{\underline{0}}

\newcommand{\ba}{\mathfrak{a}}
\newcommand{\bb}{\mathfrak{b}}
\newcommand{\bc}{\mathfrak{c}}
\newcommand{\bff}{\mathfrak{f}}
\newcommand{\bm}{\mathfrak{m}}
\newcommand{\bn}{\mathfrak{n}}
\newcommand{\bp}{\mathfrak{p}}
\newcommand{\bq}{\mathfrak{q}}
\newcommand{\bt}{\mathfrak{t}}
\newcommand{\fra}{\mathfrak{a}}
\newcommand{\frb}{\mathfrak{b}}
\newcommand{\frc}{\mathfrak{c}}
\newcommand{\frf}{\mathfrak{f}}
\newcommand{\frm}{\mathfrak{m}}
%\renewcommand{\frm}{\mathfrak{m}}
\newcommand{\frn}{\mathfrak{n}}
\newcommand{\frp}{\mathfrak{p}}
\newcommand{\frq}{\mathfrak{q}}
\newcommand{\frt}{\mathfrak{t}}

\newcommand{\bA}{\mathbb{A}}
\newcommand{\bB}{\mathbb{B}}
\newcommand{\bC}{\mathbb{C}}
\newcommand{\bD}{\mathbb{D}}
\newcommand{\bE}{\mathbb{E}}
\newcommand{\bF}{\mathbb{F}}
\newcommand{\bG}{\mathbb{G}}
\newcommand{\bH}{\mathbb{H}}
\newcommand{\bI}{\mathbb{I}}
\newcommand{\bJ}{\mathbb{J}}
\newcommand{\bK}{\mathbb{K}}
\newcommand{\bL}{\mathbb{L}}
\newcommand{\bM}{\mathbb{M}}
\newcommand{\bN}{\mathbb{N}}
\newcommand{\bO}{\mathbb{O}}
\newcommand{\bP}{\mathbb{P}}
\newcommand{\bQ}{\mathbb{Q}}
\newcommand{\bR}{\mathbb{R}}
\newcommand{\bS}{\mathbb{S}}
\newcommand{\bT}{\mathbb{T}}
\newcommand{\bU}{\mathbb{U}}
\newcommand{\bV}{\mathbb{V}}
\newcommand{\bW}{\mathbb{W}}
\newcommand{\bX}{\mathbb{X}}
\newcommand{\bY}{\mathbb{Y}}
\newcommand{\bZ}{\mathbb{Z}}


\newcommand{\al}{\alpha}
\newcommand{\be}{\beta}
\newcommand{\ga}{\gamma}
\newcommand{\de}{\delta}
\newcommand{\pa}{\partial}   %pretend its Greek
\newcommand{\epz}{\varepsilon}
\newcommand{\ph}{\phi}
\newcommand{\phz}{\varphi}
\newcommand{\et}{\eta}
\newcommand{\io}{\iota}
\newcommand{\ka}{\kappa}
\newcommand{\la}{\lambda}
\newcommand{\tha}{\theta}
\newcommand{\thz}{\vartheta}
\newcommand{\rh}{\rho}
\newcommand{\si}{\sigma}
\newcommand{\ta}{\tau}
\newcommand{\ch}{\chi}
\newcommand{\ps}{\psi}
\newcommand{\ze}{\zeta}
\newcommand{\om}{\omega}
\newcommand{\GA}{\Gamma}
\newcommand{\LA}{\Lambda}
\newcommand{\DE}{\Delta}
\newcommand{\SI}{\Sigma}
\newcommand{\THA}{\Theta}
\newcommand{\OM}{\Omega}
\newcommand{\XI}{\Xi}
\newcommand{\UP}{\Upsilon}
\newcommand{\PI}{\Pi}
\newcommand{\PS}{\Psi}
\newcommand{\PH}{\Phi}

\newcommand{\com}{\circ}     % composition of functions
\newcommand{\iso}{\simeq}    % preferred isomorphism symbol
\newcommand{\ten}{\otimes}   % tensor product
\newcommand{\add}{\oplus}    % direct sum

\newcommand{\ul}{\underline}
\newcommand{\nsubset}{\not\subset}
\newcommand{\tld}{\widetilde }
\renewcommand{\:}{\colon}


\newcommand{\rtarr}{\longrightarrow}
\newcommand{\ltarr}{\longleftarrow}
\newcommand{\from}{\longleftarrow}
\newcommand{\monoto}{\lhook\joinrel\relbar\joinrel\rightarrow}
\newcommand{\epito}{\relbar\joinrel\twoheadrightarrow}

%%%%%%%%% math short hand
%%%% gothic
\newcommand{\Schs}{\mathfrak S\mathfrak c\mathfrak h_{S}}
\newcommand{\LocFrees}{\mathfrak L\mathfrak o\mathfrak c\mathfrak F\mathfrak
 r\mathfrak e\mathfrak e_{S}}
\newcommand{\A}{\mathfrak A}
\newcommand{\Ab}{\mathfrak A\mathfrak b}
\newcommand{\B}{\mathfrak B}
\newcommand{\M}{\mathfrak M\mathfrak o\mathfrak d}
\newcommand{\Mg}{\mathfrak M_g}
\newcommand{\Mgbar}{\overline{\mathfrak M}_g}
\newcommand{\Mh}{\mathfrak M_h}
\newcommand{\Mhbar}{\overline{\mathfrak M}_h}
\newcommand{\maxm}{\mathfrak m}

%%%% curly
\newcommand{\m}{\scr M}
\newcommand{\n}{\scr N}
\newcommand{\cO}{\mathcal O}
\renewcommand{\O}{\mathcal O}
\newcommand{\f}{\scr F}
\renewcommand{\O}{\scr O}
\newcommand{\I}{\scr I}
\newcommand{\J}{\scr{J}}

%%%% Blackboard bold
\newcommand{\C}{\mathbb {C}}
\newcommand{\N}{\mathbb {N}}
\newcommand{\R}{\mathbb {R}}
\newcommand{\PP}{\mathbb {P}}
\newcommand{\Z}{\mathbb {Z}}
\newcommand{\Q}{\mathbb {Q}}
\renewcommand{\r}{\mathbb R^{+}}
\newcommand{\NZ}{\mbox{$\mathbb{N}$}}
\renewcommand{\O}{\mbox{$\mathcal{O}$}}
\renewcommand{\P}{\mathbb{P}}
\newcommand{\ZZ}{\mbox{$\mathbb{Z}$}}
%%%%
\newcommand{\infinity}{\infty}
\newcommand{\ney}{\overline{NE}(Y)}
\newcommand{\nex}{\overline{NE}(X)}
\newcommand{\nes}{\overline{NE}(S)}
%%%%
\newcommand{\sub}{\subseteq}
\newcommand{\ratmap}{\dasharrow}
\newcommand{\eq}{\equiv}
\newcommand{\myquad}{\ }
%%%
%%%%%%% operators
\DeclareMathOperator{\Char}{{char}}
\DeclareMathOperator{\Cart}{{Cartier}}
\DeclareMathOperator{\fpt}{{fpt}}
\DeclareMathOperator{\lct}{{lct}}
\DeclareMathOperator{\divisor}{{div}}
\DeclareMathOperator{\Div}{{div}}
\DeclareMathOperator{\ord}{{ord}}
\DeclareMathOperator{\Frac}{{Frac}}
\DeclareMathOperator{\Ann}{{Ann}}
\DeclareMathOperator{\rd}{{rd}}
\DeclareMathOperator{\an}{{an}}
\DeclareMathOperator{\height}{{ht}}
\DeclareMathOperator{\exc}{{exc}}
\DeclareMathOperator{\coherent}{{coh}}
\DeclareMathOperator{\quasicoherent}{{qcoh}}
\DeclareMathOperator{\sn}{{sn}}
\DeclareMathOperator{\wn}{{wn}}
\DeclareMathOperator{\id}{{id}}
\DeclareMathOperator{\codim}{codim}
\DeclareMathOperator{\coker}{{coker}}
%%\DeclareMathOperator{\ker}{{ker}}
\DeclareMathOperator{\im}{{im}}
\DeclareMathOperator{\sgn}{{sgn}}
%%\DeclareMathOperator{\hom}{{Hom}}
\DeclareMathOperator{\opp}{{op}}
\DeclareMathOperator{\ext}{{Ext}}
\DeclareMathOperator{\Tor}{{Tor}}
\DeclareMathOperator{\pic}{{Pic}}
\DeclareMathOperator{\pico}{{Pic}^{\circ}}
\DeclareMathOperator{\aut}{{Aut}}
\DeclareMathOperator{\bir}{{Bir}}
\DeclareMathOperator{\lin}{{Lin}}
\DeclareMathOperator{\sym}{{Sym}}
\DeclareMathOperator{\rank}{{rank}}
\DeclareMathOperator{\rk}{{rk}}
\DeclareMathOperator{\pgl}{{PGL}}
\DeclareMathOperator{\gl}{{GL}}
\DeclareMathOperator{\Gr}{{Gr}}
\DeclareMathOperator{\ob}{{Ob}}
\DeclareMathOperator{\mor}{{Mor}}
\DeclareMathOperator{\supp}{{supp}}
\DeclareMathOperator{\Supp}{{Supp}}
\DeclareMathOperator{\Sing}{{Sing}}
\DeclareMathOperator{\var}{{Var}}
\DeclareMathOperator{\Spec}{{Spec}}
\DeclareMathOperator{\Proj}{{Proj}}
\DeclareMathOperator{\Tot}{{Tot}}
\DeclareMathOperator{\Aut}{Aut}
\DeclareMathOperator{\Lef}{Lef}
\DeclareMathOperator{\wt}{wt}
\DeclareMathOperator{\twoRC}{{RC_2^n}}
\DeclareMathOperator{\ptRC}{{RC_{\bullet}}}
\DeclareMathOperator{\twoptRC}{{RC^2_{\bullet}}}
\DeclareMathOperator{\Univ}{Univ}
\DeclareMathOperator{\Univrc}{{Univ^{rc}}}
\DeclareMathOperator{\twoUnivrc}{{Univ^{rc, 2}}}
\DeclareMathOperator{\ptUnivrc}{{Univ^{rc}_{\bullet}}}
\DeclareMathOperator{\twoptUnivrc}{{Univ_{\bullet}^{rc, 2}}}
\DeclareMathOperator{\charact}{char}
\DeclareMathOperator{\Chow}{Chow}
\DeclareMathOperator{\Dubbies}{Dubbies^n}
\DeclareMathOperator{\Ext}{Ext}
\DeclareMathOperator{\Hilb}{Hilb}
\DeclareMathOperator{\Hom}{Hom}
\DeclareMathOperator{\sHom}{{\sH}om}
\DeclareMathOperator{\Hombir}{Hom_{bir}^n}
\DeclareMathOperator{\Image}{Image}
\DeclareMathOperator{\genus}{genus}
\DeclareMathOperator{\Imaginary}{Im}
\DeclareMathOperator{\Img}{Im}
\DeclareMathOperator{\Ker}{Ker}
\DeclareMathOperator{\locus}{locus}
\DeclareMathOperator{\Num}{Num}
\DeclareMathOperator{\Pic}{Pic}
\DeclareMathOperator{\RatCurves}{RatCurves^n}
\DeclareMathOperator{\RC}{RatCurves^n}
\DeclareMathOperator{\red}{red}
\DeclareMathOperator{\Reg}{Reg}
\DeclareMathOperator{\psl}{PGL}
\DeclareMathOperator{\mult}{mult}
\DeclareMathOperator{\mld}{mld}
\renewcommand{\mod}[1]{\,(\textnormal{mod}\,#1)}
%%%%%%%%%%%%%%%%%%%%%%%%%%%%%%%%%%%%%
\def\spec#1.#2.{{\bold S\bold p\bold e\bold c}_{#1}#2}
\def\proj#1.#2.{{\bold P\bold r\bold o\bold j}_{#1}\sum #2}
\def\ring#1.{\scr O_{#1}}
\def\map#1.#2.{#1 \to #2}
\def\longmap#1.#2.{#1 \longrightarrow #2}
\def\factor#1.#2.{\left. \raise 2pt\hbox{$#1$} \right/
\hskip -2pt\raise -2pt\hbox{$#2$}}
\def\pe#1.{\mathbb P(#1)}
\def\pr#1.{\mathbb P^{#1}}
\newcommand{\sheafspec}{\mbox{\bf Spec}}
\newcommand{\len}[2]{{{\bf \ell}_{#1}{\left(#2\right)}}}

%%%%%%%%%%%%%%%%%%%%%%%%%%%%%%%%%%%%%%%%%%%%%%%%%
%%%%%% cohomology and short exact sequences %%%%%
%%%%%%%%%%%%%%%%%%%%%%%%%%%%%%%%%%%%%%%%%%%%%%%%%
\def\coh#1.#2.#3.{H^{#1}(#2,#3)}
\def\dimcoh#1.#2.#3.{h^{#1}(#2,#3)}
\def\hypcoh#1.#2.#3.{\mathbb H_{\vphantom{l}}^{#1}(#2,#3)}
\def\loccoh#1.#2.#3.#4.{H^{#1}_{#2}(#3,#4)}
\def\dimloccoh#1.#2.#3.#4.{h^{#1}_{#2}(#3,#4)}
\def\lochypcoh#1.#2.#3.#4.{\mathbb H^{#1}_{#2}(#3,#4)}
%%%%%%%%%%
\def\ses#1.#2.#3.{0  \longrightarrow  #1   \longrightarrow
 #2 \longrightarrow #3 \longrightarrow 0}
\def\sesshort#1.#2.#3.{0
 \rightarrow #1 \rightarrow #2 \rightarrow #3 \rightarrow 0}
 \def\sesa#1{0
 \rightarrow #1 \rightarrow #1 \rightarrow #1 \rightarrow 0}

%\renewcommand{\to}[1][]{\xrightarrow{\ #1\ }}
\newcommand{\onto}[1][]{\protect{\xrightarrow{\ #1\ }\hspace{-0.8em}\rightarrow}}
\newcommand{\into}[1][]{\lhook \joinrel \xrightarrow{\ #1\ }}
%%%%%%%%%%
%%%%%%%%%% iff
\def\iff#1#2#3{
    \hfil\hbox{\hsize =#1 \vtop{\noin #2} \hskip.5cm
    \lower.5\baselineskip\hbox{$\Leftrightarrow$}\hskip.5cm
    \vtop{\noin #3}}\hfil\medskip}
%%%%%%%%%%%%%%%%%%%%%%%%%%%%%%
\def\myoplus#1.#2.{\underset #1 \to {\overset #2 \to \oplus}}
\def\assign{\longmapsto}
%%%%%%%%%%%%%%%%%%%%%%%%%%%%%%
%%%%%%%%%%%%%%%%%%%%%%%%%%%%%%%%%%
%%% Arrows %%%%%%%%%%%%%%
%%%%%%%%%%%%%%%%%%%%%%%%%%%%%%%%%%%

\usepackage{stmaryrd}

\usepackage{verbatim}
\usepackage{enumerate}

\DeclareMathOperator{\HH}{H}
%\DeclareMathOperator{\Image}{image}

\begin{document}
\title{{TestIdeals} package for \emph{Macaulay2}}
\author{Daniel Hern\'andez}
\author{Mordechai Katzman}
\author{Marcus Robinson}
\author{Karl Schwede}
\author{Daniel Smolkin}
\author{Pedro Teixeira}
\author{Emily Witt}
\date{\today}
\address{Department of Mathematics, University of Utah, 155 S 1400 E Room 233, Salt Lake City, UT, 84112}
\email{schwede@math.utah.edu}

\begin{abstract}
  This note describes a \emph{Macaulay2} package for computations in commutative rings prime related to $p^{-e}$-linear and $p^{e}$-linear  maps,
  singularities defined in terms of these maps,  various test ideals and modules, and ideals compatible with a given $p^{-e}$-linear map.
\end{abstract}


\subjclass[2010]{13A35}

\keywords{Macaulay2}

%\thanks{The first named author was supported in part by the NSF FRG Grant DMS \#1265261, NSF CAREER Grant DMS \#1252860 and a Sloan Fellowship.}
%\thanks{The second named author was supported in part by the NSF CAREER Grant DMS \#1252860.}
\maketitle

\section{Introduction}

This paper constructive methods for computing various objects related to commutative rings of prime characteristic $p$.
Such a ring $R$ comes equipped with a built-in endomorphism, namely the Frobenius endomorphism $f:R \rightarrow R$ which is the basis for many constructions and definitions
which affords a handle on many problems which is not otherwise available. Two notable examples of the use of the Frobenius endomorphism are the theory of tight closure 
(see \cite{HochsterHunekeTC1} for an introduction)
and the resulting theory of test ideals
(cf. the survey \cite{SchwedeTuckerTestIdealSurvey}).

Work on the \emph{TestIdeals} package started as a library of methods to compute the Frobenius roots, and $\star$-closures, and to implement the algorithm for computing parameter-test-ideals
developed in \cite{KatzmanParameterTestIdealOfCMRings} and \cite{KatzmanFrobeniusMapsOnInjectiveHulls} used to produce the examples in that paper.
Later, the algorithm in  \cite{KatzmanSchwedeAlgorithm} for computing prime ideals compatible with a given $p^{-e}$-linear map was implemented and 
used to produce the examples in that paper. The work to turn these libraries into a comprehensive package for computations related to test-ideals
started in earnest on the Summer of 2012 in the Macaulay2 workshop held at Wake Forest. Further work on this package was undertaken in the Macaulay2 workshops in 
Berkeley (2014 and 2017), and Salt Lake City (2016). 
{\hfill\large\color{red} [Expand: how were the test module methods start life?]}\\



\subsection*{Acknowledgements}

We thank ??? for useful conversations and comments on the development of this package.

\section{Frobenius powers and Frobenius roots}\label{Section: Frobenius powers and Frobenius roots}

%%% Throughout  this paper $R$ will denote a polynomial ring over a field $\mathbb{K}$ of prime characteristic $p$.

Let $S$ denote any commutative ring of prime characteristic $p$.

\begin{definition}
For any ideal $I\subseteq S$ and any integer $e\geq I$, we define the \emph{$e$th Frobenius power of $I$} to be the ideal denoted $I^{[p^e]}$ which is
generated by all $p^e$th powers of elements in $I$.
\end{definition}

It is easy to see that, if $I$ is generated by $g_1, \dots, g_\ell$, $I^{[p^e]}$ is generated by $g_1^{p^e}, \dots, g_\ell^{p^e}$.


\begin{definition}
For any ideal $I\subseteq S$ and any integer $e\geq I$, we define the \emph{$e$th Frobenius root of $I$} to be the ideal denoted $I^{[1/p^{e}]}$ which is
the smallest ideal $J$ such that $I\subseteq J^{[p^e]}$, if such ideal exists.
\end{definition}

$e$th Frobenius roots exist in polynomial rings (cf.~\cite[\S 2]{BlickleMustataSmithDiscretenessAndRationalityOfFThresholds}) and in power series rings
(cf.~\cite[\S 5]{KatzmanParameterTestIdealOfCMRings}).


\begin{verbatim}
i2 :      R=ZZ/5[x,y,z]
i3 :      I=ideal(x^6*y*z+x^2*y^12*z^3+x*y*z^18)
                18    2 12 3    6
o3 = ideal(x*y*z   + x y  z  + x y*z)
o3 : Ideal of R
i4 :      frobeniusPower(1/5,I)
                2   3
o4 = ideal (x, y , z )
\end{verbatim}

We can extend the definition of Frobenius powers as follows
\begin{definition}[cf. \cite{HernandezTeixeiraWittFrobeniusPowers}]
Let  $I\subseteq S$ be an ideal.
\begin{enumerate}
 \item[(a)] If $a$ is a positive integer with base-$p$ expansion  $a=a_0 + a_1 p +  \dots + a_r p^r$, we define
 $I^{[n]}=I^{a_0} \left(I^{a_1}\right)^{[p]} \dots  (I^{a_r})^{[p^r]}$. %\left( I^{a_r}\right)^{[p^r]}$ .
 \item[(b)] If $t$ is a non-negative rational number of the form $t = a/p^e$, we define  $I^{[t]} = (I^{[a]})^{[1/p^e]}.$
 \item[(c)] If $t$ is any non-negative rational number, and $\{a_n/p^{e_n}\}_{n\geq 1}$ is a sequence of rational numbers converging to $t$ from above, we define $I^{[t]}$
 to be the stable value of the increasing chain of ideals $\{I^{[a_n/p^{e_n}]}\}_{n\geq 1}$.

\end{enumerate}
\end{definition}


\begin{verbatim}
i5 :      frobeniusPower(1/2, ideal(y^2-x^3))
o5 = ideal 1
o5 : Ideal of R
i6 :      frobeniusPower(5/6, ideal(y^2-x^3))
o6 = ideal (y, x)
o6 : Ideal of R
\end{verbatim}



\section{$p^{-e}$- and $p^{e}$-linear maps}\label{Section: p-linear maps}

\begin{definition}%{\hfill\large\color{red} [Add references]}\\
Let  $M$ be an $S$-module and $e$ a non-negative integer.
\begin{enumerate}
 \item[(a)] A $p^{-e}$-linear map $\phi:M \rightarrow M$ is an additive map such that
 $\phi(s^{p^e} m)= s\phi(m)$ for all $s\in S$ and $m\in M$.
 \item[(b)] A $p^{e}$-linear map $\psi:M \rightarrow M$ is an additive map such that
 $\phi(s m)= s^{p^e}\phi(m)$ for all $s\in S$ and $m\in M$.
\end{enumerate}
\end{definition}


The following two examples describe two prototypical
$p^{-e}$- and $p^{e}$-linear maps.
\begin{example}
For any $S$-module $M$, we can construct a new $S$-module $M^{1/p^e}$ with elements $\{  m^{1/p^e} \,|\,m\in M\}$ by defining
$m_1^{1/p^e} +  m_2^{1/p^e} =  (m_1 +  m_2)^{1/p^e}$ for all $m_1, m_2 \in M$ and
$s  m^{1/p^e}=  (s^{p^e} m)^{1/p^e}$ for all $m\in M$ and $s\in S$.

Consider any $\phi\in \Hom_S(M^{1/p^e}, M)$: if we identify $M^{1/p^e}$ with $M$ we can interpret $\phi$ as a $p^{-e}$-linear map.
\end{example}

\begin{example}
The $e$th Frobenius map $f:S \rightarrow S$  raising elements to their $p^e$th power is clearly $p^{e}$-linear.
Furthermore, for any ideal $I\subseteq S$ and $k\geq 0$, $f$ induces a $p^{e}$-linear map $\HH_I^k (S) \rightarrow \HH_I^k (S)$.
\end{example}

Let $R$ be a polynomial ring with irrelevant ideal $\mathfrak{m}$ and let $g\in R\setminus \{0\}$.
Let $E=E_{R_{\mathfrak{m}}}(R_{\mathfrak{m}}/\mathfrak{m})$ denote the injective hull of $R_{\mathfrak{m}}/\mathfrak{m}$.

The Frobenius map on $R$ induces a Frobenius map $S$ and on
$\HH^{\dim R-1}_{\mathfrak{m}} (S)=E_S(S/\mathfrak{m}S)=\Ann_E g$ and the kernel of this map is given by
$\Ann_E (g^{p-1}R)^{[1/p]}$ (cf.~\cite[\S 5]{KatzmanParameterTestIdealOfCMRings}).

\begin{verbatim}
i2 :      R=ZZ/5[x,y,z]
i3 :      g=x^3+y^3+z^3
i4 :      u=g^(5-1)
i5 :      frobeniusPower(1/5,ideal(u))
o5 = ideal (z, y, x)
o5 : Ideal of R

i6 :      R=ZZ/7[x,y,z]
i7 :      g=x^3+y^3+z^3
i8 :      u=g^(7-1)
i9 :      frobeniusPower(1/7,ideal(u))
o9 = ideal 1
\end{verbatim}

Thus we see that the induced $p^{e}$-linear map on $\HH_{(x,y,z)}^{2} \left( \mathbb{K}[x,y,z]/(x^3+y^3+z^3) \right)$ is injective
when the characteristic of $\mathbb{K}$ is $7$ and non-injective when the characteristic is $5$.

\subsection{Implementation and complexity}

The computation of Frobenius roots $(-)^{[1/p^e]}$ is the computational workhorse behind many of the methods in \emph{TestIdeals}
hence it is important to understand how it is computed and its computational complexity. 

Let $R=\mathbb{K}[x_1, \dots, x_n]$ and fix $B$ to be $\mathbb{K}$-basis for $\mathbb{K}^{1/p^e}$;
it is not hard to see that $R^{1/p^e}$ is a free $R$-module with free basis
$$\left\{ b x_1^{\alpha_1/p^e} \dots x_n^{\alpha_n/p^e} \,|\, 0\leq \alpha_1, \dots, \alpha_n < p^e \right\}.$$

\begin{proposition}(cf. \cite[\S 5]{KatzmanParameterTestIdealOfCMRings})
\begin{enumerate}
\item[(a)] If the ideal $I\subseteq R$ is generated by $g_1, \dots, g_k\in R$, then $I^{[1/p^e]}=\sum_{i=1}^k (g_i R)^{[1/p^e]}$.
\item[(b)] If 
\begin{equation}\label{eqn1}
g=\bigoplus_{b\in B, \atop{0\leq \alpha_1, \dots, \alpha_n < p^e}} r_{b, \alpha_1, \dots, \alpha_n} b x_1^{\alpha_1/p^e} \dots x_n^{\alpha_n/p^e}, 
\end{equation}
then $(gR)^{[1/p^e]}$ is generated by $\{ r_{b, \alpha_1, \dots, \alpha_n} \,|\, b\in B, 0\leq \alpha_1, \dots, \alpha_n < p^e \}$.
\end{enumerate}
\end{proposition}

Thus the computation of Frobenius roots of ideals reduces to the case of principal ideals, and the complexity grows linearly with the number of generators of the ideal.
Furthermore the calculation of  $(gR)^{[1/p^e]}$ reduces to finding the summands in (\ref{eqn1}), which essentially amounts 
taking each term in $g$, computing the $p^e$th root of the coefficient in $\mathbb{K}$, and dividing the monomial exponent vector by $p^e$ with remainder.
Thus the complexity of  computing $(gR)^{[1/p^e]}$ is proportional to the number of terms in $g$ and is independent of its degree.
\emph{The calculation of Frobenius roots does not involve the calculation of Gr\"obner bases.}



\section{$F$-singularities}

This package includes a method for checking if a ring of finite type over a prime field is $F$-injective or not.
\begin{definition}
A local ring $(R, \mathfrak{m})$ is called \emph{F-injective} if the map $H^{i}_{\mathfrak{m}}(R) \rightarrow H^{i}_{\mathfrak{m}}(R^{1/p})$ is injective for all $i >0$. An arbitrary ring is called $F$-injective if each of the localizations at a prime ideal are $F$-injective.
\end{definition}

\begin{verbatim}
i2 : R = ZZ/7[x,y,z]/ideal(x^3 + y^3 + z^3);
i3 : isFinjective(R)
o3 = true
i4 : R = ZZ/5[x,y,z]/ideal(x^3 + y^3 + z^3);
i5 : isFinjective(R)
o5 = false
\end{verbatim}

The algorithm {\tt isFinjective} determines whether the ring  $R = S/I$ is F-injective or not. The algorithm works by checking the injectivity of the frobenius map using the functoriallity of $\Ext$.  The algorithm starts by computing  the map $R \rightarrow F_{*}R$ using {\tt frobPFMap}. This outputs a map represented as a matrix over $R$, using {\tt pushFwdToAmbient} to allows this map to instead be represented over the ambient ring $S$. The next step computes the module $\Ext^{i}( \blank, S)$ using the map from the previous step. Finally the algorithm checks the dimension of the cokernel of  $\Ext^{i}( \blank, S)$. If the dimension does not equal negative one then the Frobenius action is not injective and the algorithm terminates and returns false. Otherwise the algorithm continues on and checks the next degree in the same way. 

The {\tt CanonicalStrategy} tag can be used to modify the strategy the algorithm uses to check the Frobenius action on the top local cohomology. By default the algorithm is set to {\tt CanonicalStrategy => Katzman} which then uses the strategy of Katzman  {\hfill\large\color{red} [Add references]}. If the tag is set to anything else {\tt CanonicalStrategy => null} the algorithm checks the top local cohomology using the same brute force strategy used to check the injectivity at lower degrees. The Katzman strategy is typically much faster. 

There are a number of options to improve the performance of the algorithm if the ring of interest is nice enough. If the ring is Cohen-Macaulay then setting {\tt AssumeCM => true} lets the algorithm check the Frobenius action only on top cohomology (which is typically much faster).  The default value is {\tt false}. Of course, telling the algorithm to assume the ring is Cohen-Macaulay when it is not can lead to an incorrect answer if the non-injective Frobenius occurs in a lower degree. For an example of this see the documentation.  If the ring is reduced then setting {\tt AssumedReduced => true} avoids computing the bottom local cohomology, if the ring is normal then setting {\tt AssumeNormal => true} avoids computing the bottom two local cohomologies. The default setting for both of these tags is {\tt false}. 

By default the algorithm checks for $F$-injectivity everywhere however one can choose to check $F$-injectivity only at the origin by setting the option {\tt IsLocal => true}. 
\begin{verbatim}
i2 : R = ZZ/5[x,y,z]/ideal( (x-1)^4 + y^4 + z^4 );
i3 : isFinjective(R)
o3 = false
i4 : isFinjective(R, IsLocal=>true)
o4 = true
\end{verbatim}


\begin{definition}
A ring $R$ is called strongly $F$-regular if $\tau(R) = R$. 
\end{definition}

The command {\tt isFregular} checks whether a ring or pair is strongly $F$-regular. Below are two examples one $F$-regulare and one not.

\begin{verbatim}
i2 : R = ZZ/5[x,y,z]/ideal(x^2 + y*z);
i3 : isFregular(R)
o3 = true
i4 : R = ZZ/7[x,y,z]/ideal(x^3 + y^3 + z^3);
i5 : isFregular(R)
o5 = false
\end{verbatim}

We can also perform these types computations for pairs.

\begin{verbatim}
i2 : R = ZZ/5[x,y];
i3 : f = y^2-x^3;
i4 : isFregular(1/2, f)
o4 = true
i5 : isFregular(5/6, f)
o5 = false
i6 : isFregular(4/5, f)
o6 = false
i7 : isFregular(4/5-1/100000, f)
o7 = true
\end{verbatim}

The inputs for {\tt isFregular} can be a ring, an integer and a ring element, a rational number and a ring element or a list of rational numbers and a list of ring elements. 

If the input ring is $\mathbb{Q}$-Gorenstein then in each of the cases above the output is a boolean indicating if the ring is strongly $F$-regular. If the input ring is not $\mathbb{Q}$-Gorenstein then the algorithm can be used to determine if a ring is strongly $F$-regular but cannot prove a ring is not strongly $F$-regular

In the case that $R$ is $\mathbb{Q}$-Gorenstein, the algorithm works by computing the test ideal of the ring (or the pair) and then uses {\tt isSubset} to check if $(1)$  is contained in the test ideal. In the non-$\mathbb{Q}$-Gorenstein case the algorithm checks for strong $F$-regularity by checking if $(1)$ is contained in better and better approximations of the test ideal. To compute approximations of the test ideal the algorithm uses {\tt frobeniusRoot} to compute the $e$th root of $c(I^{[p^{e}]} : I)$ (in the case we are checking strong F-regularity for a ring, appropriate modifications are made for pairs). If at any step $(1)$ is contained in the approximation then then the algorithm returns {\tt true}. Otherwise the algorithm continues checking for each $e$ until a specified limit is reached. The default limit is 2 and can be changed using {\tt DepthOfSearch => ZZ}.  


The default behavior of {\tt isFregular} checks for strong $F$-regularity everywhere. If the option {\tt IsLocal => true}, the algorithm will only check at the origin (this uses a similar computation but checks for the containment of $(1)$ instead of the test ideal plus the maximal ideal at the origin). Below is an example for both a ring and a pair. 

\begin{verbatim}
i2 : R = ZZ/7[x,y,z]/ideal((x-1)^3+(y-2)^3+z^3);
i3 : isFregular(R)
o3 = false
i4 : isFregular(R, IsLocal=>true)
o4 = true
i5 : R = ZZ/13[x,y];
i6 : f = (y-2)^2 - (x-3)^3;
i7 : isFregular(5/6, f)
o7 = false
i8 : isFregular(5/6, f, IsLocal=>true)
o8 = true
\end{verbatim}


\section{Test ideals}

In this section, we explain how to compute parameter test modules, parameter test ideals, test ideals and HSLG-modules\footnote{HSLG-modules can be used to give a scheme structure to the $F$-injective or $F$-pure locus.}.

\subsection{Parameter test modules}

Given an $F$-finite reduced ring $R$, the Frobenius map $R \to R^{1/p^e}$ is dual to $T : \omega_{R^{1/p^e}} \to \omega_R$.  As before in \autoref{}, we can represent $\omega_R \subseteq R$ as an ideal, we can write $R = S/I$, and so we can find a $u \in S^{1/p^e}$ representing the map $\omega_{R^{1/p^e}} \to \omega_R$ \cite{}.  See also \autoref{} below.

The parameter test submodule is the smallest submodule $M \subseteq \omega_R$ (and hence ideal of $R$ since $M \subseteq R$) that agrees generically with $\omega_R$ and that satisfies
\[
T (M^{1/p^e}) \subseteq M.
\]
From within Macualay2, we can compute this using the {\tt testModule} command as follows.
\begin{verbatim}
i3 : R = ZZ/5[x,y,z]/ideal(x^4+y^4+z^4);
i4 : N = testModule(R);
i5 : N#0
             2             2        2
o5 = ideal (z , y*z, x*z, y , x*y, x )
o5 : Ideal of R
i6 : N#1
o6 = ideal 1
o6 : Ideal of R
\end{verbatim}
The output of {\tt testModule} is a list with three items.  The first entry is the test module, the second is the canonical module (as an ideal) that it lives inside, and the third is the element $u$ described above (not listed here, since it is rather complicated).  Note since this ring is Gorenstein, the canonical module is simply represented as the unit ideal.

Here is another example where the ring is not Gorenstein.
\begin{verbatim}
i2 : R = ZZ/5[x,y,z]/ideal(y*z, x*z, x*y);
i3 : N = testModule(R);
i4 : N#0
             2   2   2
o4 = ideal (z , y , x )
o4 : Ideal of R
i5 : N#1
o5 = ideal (y - z, x + z)
o5 : Ideal of R
\end{verbatim}

We briefly explain how this is computed.  First we find a \emph{test element}.
\begin{remark}[Computation of test elements]
\label{rem.ComputationOfTestElements}
Roughly, we recall that an element of the Jacobian ideal that is not contained in any minimal prime is a test element \cite{}.  We compute test elements by computing random partial derivatives (and linear combinations thereof) until we find an element that doesn't vanish at all the minimal primes.  This method is much faster than computing the entire Jacobian ideal.  If you know your ring is a domain, you can use the {\tt AssumeDomain} flag to speed this up further.
\end{remark}

After the test element $c$ has been identified, we pullback the ideal $\omega_R$ to an ideal $J = S$.  Then we compute the following ascending sequence of ideals
\begin{equation}
\label{eq.AscendIdealExplanation}
\begin{array}{rll}
J_1 := & = J + (c J)^{[1/p]} & = {\tt J + frobeniusRoot(1, c*J)}\\
J_2 := & = J + (c J)^{[1/p]} + (c^{1+p} J)^{[1/p^2]} & = {\tt J + frobeniusRoot(1, c*J_1)}\\
J_3 := & = J + (c J)^{[1/p]} + (c^{1+p} J)^{[1/p^2]} + (c^{1+p+p^2} J)^{[1/p^3]} & = {\tt J + frobeniusRoot(1, c*J_2)}.\\
\dots
\end{array}
\end{equation}
As soon as this ascending sequence of ideals stabilizes, we are done.
In fact, because this strategy is used in several contexts, you can call it directly for a chosen ideal $J$ and $u$ with the function {\tt ascendIdeal} (this is done for test ideals below).

We can also compute parameter test modules of pairs $\tau(\omega_R, f^{t})$ with $t \in \mathbb{Q}_{\geq 0}$.  This is done by modifying the element $u$ when the denominator of $t$ is not divisible by $p$.  When $t$ has $p$ in the denominator, we rely on the fact (see \cite{}) that
\[
\begin{array}{rcl}
T(\tau(\omega_R, f^a)) & = & \tau(\omega_R, f^{a/p})\\
{\tt frobeniusRoot(1, u*I_1)} & {\tt =} & {\tt I_2}
\end{array}
\]
Where the second line roughly explains how this is accomplished internally, here ${\tt I_1}$ is $\tau(\omega_R, f^a)$ pulled back to $S$ and likewise $I_2$ defines $\tau(\omega_R, f^a)$ modulo the defining ideal of $R$.

\begin{remark}[Optimizations in {\tt ascendIdeal} and other {\tt testModule} computations]
Throughout the computations described above, we very frequently use the following fact:
\[
(f^p \cdot J)^{[1/p]} = f \cdot (J^{[1/p]}).
\]
To access this enhancement, one should try to pass functions like {\tt ascendIdeal} and {\tt frobeniusRoot} the elements and their exponents (see the documentation).  In particular, do not multiply out things like $f^n \cdot J$ if you will be applying {\tt frobeniusRoot}.
\end{remark}

\subsection{Parameter test ideals}

The parameter test ideal is simply the annihilator of $\omega_R/\tau(\omega_R)$.  Or in other words, it is
\[
\tau(\omega_R) : \omega_R.
\]
This can also be described as
\[
\bigcap_{I} (I^* : I)
\]
where $I$ varies over ideals defined by a partial system of parameters and $I^*$ denotes its tight closure.  This latter description is not computable however.

\begin{verbatim}
todo (Moty will insert a good example)
\end{verbatim}

\subsection{Test ideals}

For an $F$-finite reduced ring $R = S/I$ where $S$ is a regular ring, the test ideal of $R$ is the smallest ideal $J$ in $R$, not contained in any minimal prime, such that for all $e > 0$ and all
\[
\phi \in \Hom_R(R^{1/p^e}, R) \text{ we have } \phi(J^{1/p^e}) \subseteq J.
\]
In the case that $R$ is Gorenstein, there is a map $\Phi_e \in \Hom_R(R^{1/p^e}, R)$ which generates the $\Hom$-set as an $R^{1/p^e}$-module.  In fact this $\Phi_e$ is identified with the map $T$ above based on the identification $\omega_R \cong R$, \cite{}.  More generally, if $R$ is $\bQ$-Gorenstein with index not divisible by $p$, then for at least sufficiently divisible $e > 0$, such a generating $\Phi_e$ still exists \cite{}.

If such a $\Phi_e$ exists, it can be identified as the generator of the module $(I^{[p^e]} : I) / I^{[p^e]}$ by Fedder's Lemma, \cite{}.  Hence we can find a corresponding\footnote{This is done via the function {\tt QGorensteinGenerator}. } $u \in I^{[p^e]} : I$.  In this case, if $c \in S$ is the pre-image of a test element of $R$, then setting $I_0 = cR$, it follows that
\[
\tau(R) = {\tt ascendIdeal(e, u, I_0)}.
\]
Recall that {\tt ascendIdeal} is explained above in \autoref{eq.AscendIdealExplanation}.  The $e$ here means all $e$th roots are taken as multiples of $e$.  See \cite{}.

Here is an example (a $\bZ/3\bZ$-quotient where $3 | (7-1)$), where exactly this logic occurs.
\begin{verbatim}
i3 : T = ZZ/7[x,y];
i4 : S = ZZ/7[a,b,c,d];
i5 : f = map(T, S, {x^3, x^2*y, x*y^2, y^3});
i6 : I = ker f;
i7 : R= S/I;
i8 : testIdeal(R)
o8 = ideal 1
o8 : Ideal of R
\end{verbatim}
However, the term $u$ constructed above can be quite involved if $e > 1$ (which happens exactly when $(p -1)K_R$ is not Cartier).  For instance, even in the above example:
\begin{verbatim}
i9 : toString(QGorensteinGenerator(1, R))
o9 = a^2*b^6*c^12+a^3*b^4*c^13+a^3*b^5*c^11*d+a^4*b^3*c^12*d+a^5*b*c^13*d+b^12
      *c^6*d^2+a^3*b^6*c^9*d^2+a^4*b^4*c^10*d^2+a^5*b^2*c^11*d^2+a^6*c^12*d^2+b
      ^13*c^4*d^3+a*b^11*c^5*d^3+a^2*b^9*c^6*d^3+a^4*b^5*c^8*d^3+a^5*b^3*c^9*d^
      3+a^6*b*c^10*d^3+a*b^12*c^3*d^4+a^2*b^10*c^4*d^4+a^3*b^8*c^5*d^4+a^4*b^6*
      c^6*d^4+a^5*b^4*c^7*d^4+a^6*b^2*c^8*d^4+a^7*c^9*d^4+a*b^13*c*d^5+a^2*b^11
      *c^2*d^5+a^3*b^9*c^3*d^5+a^4*b^7*c^4*d^5+a^5*b^5*c^5*d^5+a^6*b^3*c^6*d^5+
      a^7*b*c^7*d^5+a^2*b^12*d^6+a^3*b^10*c*d^6+a^4*b^8*c^2*d^6+a^5*b^6*c^3*d^6
      +a^6*b^4*c^4*d^6+a^7*b^2*c^5*d^6+a^8*c^6*d^6+a^4*b^9*d^7+a^5*b^7*c*d^7+a^
      6*b^5*c^2*d^7+a^7*b^3*c^3*d^7+a^8*b*c^4*d^7+a^6*b^6*d^8+a^7*b^4*c*d^8+a^8
      *b^2*c^2*d^8+a^9*c^3*d^8+a^8*b^3*d^9+a^9*b*c*d^9+a^10*d^10
\end{verbatim}
Therefore, we use a different strategy if either $(p-1)K_R$ is not Cartier or more generally if $R$ is $\bQ$-Gorenstein of index divisible by $p$ which appears to typically be faster.  We rely on the following observation, see \cite{}.
\[
\tau(\omega_R, K_R) \cong \tau(R).
\]
In fact, by embedding $\omega_R \subseteq R$, we can compute $g$ so that $\tau(\omega_R, K_R) = g \tau(R)$.  We can therefore find $\tau(R)$ if we can find $\tau(\omega_R, K_R)$.
Next, if $K_R$ is $\bQ$-Cartier with $nK_R = \Div_R(f)$ for some $f \in R$ and $n > 0$, then
\[
\tau(\omega_R, K_R) =\tau(\omega_R, f^{1/n}).
\]
Thus we directly compute $\tau(\omega_R, f^{1/n})$ via the command {\tt testModule(1/n, f)}.  Consider the following example, a $\mu_3$-quotient.
\begin{verbatim}
i2 : T = ZZ/3[x,y];
i3 : S = ZZ/3[a,b,c,d];
i4 : f = map(T, S, {x^3, x^2*y, x*y^2, y^3});
i5 : I = ker f;
i6 : R = S/I;
i7 : testIdeal(R)
o7 = ideal 1
o7 : Ideal of R
\end{verbatim}
Which uses the logic described above.

\begin{remark}[Non graded caveats]
It frequently can happen that $(I^{[p^e]} : I)/I^{[p^e]}$ is principal but Macaulay2 cannot identify the principal generator (since Macaulay2 cannot always find minimal generators of non-graded ideals or modules).  The same thing can happen when computing the $u$ corresponding to the map $T : \omega_{R^{1/p}} \to \omega_R$, as described in \autoref{} above.  In such situations, instead of a single $u$, we Macaulay2 will produce $u_1, \dots, u_n$ (all multiples of $u$, and $u$ is a linear combination of the $u_i$).  Instead of computing things like
\[
(u \cdot J)^{[1/p]},
\]
we will instead compute
\[
(u_1 \cdot J)^{[1/p]} + \dots + (u_n \cdot J)^{[1/p]}
\]
which will produce the same answer.
\end{remark}

We can similarly use the {\tt testIdeal} command to compute test ideals of pairs $\tau(R, f^t)$, and even more general mixed test ideals $\tau(R, f_1^{t_1} \cdots f_n^{t_n})$.

\subsection{HSLG module, computing $F$-pure submodules of rank-1 Cartier modules}

Again consider the maps $T^e : \omega_{R^{1/p^e}} \to \omega_R$ we have considered throughout this section.  It is a theorem of Hartshorne-Speiser, Lyubeznik, and Gabber \cite{} that the descending images
\[
\omega_R \supseteq T(\omega_{R^{1/p}}) \supseteq \dots \supseteq T^e(\omega_{R^{1/p^e}}) \supseteq T_{e+1}(\omega_{R^{1/p^{e+1}}}) \supseteq \dots
\]
stabilize for $e \gg 0$.  The function {\tt HSLGModule} computes this.
\begin{verbatim}
i4 : R = ZZ/3[x,y,z]/ideal(x^3+y^4+z^5);
i5 : L = HSLGModule(R);
i6 : L#0
                          2        2   3
o6 = ideal (0, y*z, x*z, y , x*y, x , z )
o6 : Ideal of R
i7 : L#1
o7 = ideal 1
o7 : Ideal of R
i8 : L#3
o8 = 1
\end{verbatim}
The function returns a list.  The first entry is the submodule.  The second is the canonical module.  The third entry is the $u$ representing the map on the canonical module (as in \autoref{} above) and the final entry is at what value of $e > 0$ the image stabilizes.

In a Gorenstein ring, this submodule coincides philosophically with the maximal non-LC ideal of Fujino-Schwede-Takagi.   

More generally for any ideal $J$ with a map $\phi : J^{1/p^e} \to J$, we have that the images
\[
J \supseteq \phi(J^{1/p^e}) \supseteq \phi^2(J^{1/p^{2e}}) \supseteq \dots 
\]
stabilize as well (in fact, the analogous result even holds for modules).  


\section{Ideals compatible with given $p^{-e}$-linear map}

Throughout this section let $R$ denote a polynomial $\mathbb{K}[x_1, \dots, x_n]$. In this section we address the following question:
given a $p^{-e}$ linear map $\phi: R \rightarrow R$, what are all ideals $I\subseteq R$ such that $\phi(I)=I$?
We call these ideals $\phi$-compatible.

Recall that in section \ref{Section: p-linear maps} we identified $p^{-e}$ linear maps
with elements of $\Hom_R( R^{1/p^e}, R)$ and it turns out that this is a principal $R^{1/p^e}$-module generated
the \emph{trace map} $\pi\in \Hom_R(^{1/p^e}, R)$, constructed as follows (cf. \cite[Lemma 1.6]{FedderFPureRat} and \cite[Example 1.3.1]{BrionKumarFrobeniusSplitting}).
Recall that if $B$ is a $\mathbb{K}$-basis for $\mathbb{K}^{1/p^e}$;
$R^{1/p^e}$ is a free $R$-module with basis
$$\left\{ b x_1^{\alpha_1/p^e} \dots x_n^{\alpha_n/p^e} \,|\, 0\leq \alpha_1, \dots, \alpha_n < p^e \right\} ;$$
the trace map $\pi$ is the projection onto the free summand
$R x_1^{(p^e-1)/p^e} \dots x_n^{(p^e-1)/p^e}\cong R$.

We can now write our given $\phi$ as multiplication by some $u^{1/p^e}$ followed by $\pi$ and it is not hard to see that
an ideal $I\subseteq R$ is $\phi$-compatible if and only if $u I \subseteq I^{[p^e]}$.

\begin{theorem}\label{Theorem: finitely many compatible primes}
If $\phi$ is surjective, there are finitely many $\phi$-compatible ideals, consisting of all possible intersections
of $\phi$-compatible prime ideals (cf. \cite{KumarMehtaFiniteness}, \cite{SchwedeFAdjunction},
\cite{SharpGradedAnnihilatorsOfModulesOverTheFrobeniusSkewPolynomialRing}, \cite{EnescuHochsterTheFrobeniusStructureOfLocalCohomology}).

In general, there are finitely many $\phi$-compatible prime ideals not containing $\sqrt{\Image \phi}$ (cf. \cite{KatzmanSchwedeAlgorithm}).

\end{theorem}

The method \emph{compatibleIdeals} produces the finite set of $\phi$-compatible prime ideals in the second statement of Theorem \ref{Theorem: finitely many compatible primes}.

\begin{verbatim}
i2 :      R = ZZ/3[u,v];
i3 :      u = u^2*v^2;
i4 :      compatibleIdeals(u)
o4 = {ideal v, ideal (v, u), ideal u}
o4 : List
\end{verbatim}

The defining condition $u I = I^{[p]}$ for $\phi$-compatible ideals allows us to
think of these in a dual form: write $\mathfrak{m}=(x_1, \dots, x_n)R$,
$E=E_{R\mathfrak{m}}(R_{\mathfrak{m}}/\mathfrak{m})=\HH^n_{\mathfrak{m}} (R)$, and let $T: E \rightarrow E$ be the $p^e$-linear map
induced from the Frobenius map on $R$.
If $\psi=u T$, then $\psi \Ann_E I \subseteq \Ann_E I$ if and only if $u I = I^{[p]}$!
Thus finding all $R$-submodules of $E$ compatible with $\psi=u T$ also amounts to finding all
$\phi=\pi \circ u^{1/p^e}$-compatible ideals.


Let $\mathbb{K}=\mathbb{Z}/2\mathbb{Z}$, $R=\mathbb{K}[x_1, x_2, x_3, x_4, x_5]$, $\mathfrak{m}=(x_1, \dots, x_5)R$, and let $I$ be the ideal of
$2\times 2$ minors of
$$
\left[
\begin{array}{c c c c}
 x_1 & x_2 & x_2 & x_5\\
 x_4 & x_4 & x_3 & x_1
\end{array}
\right]
$$
In (cf.~\cite[\S 9]{KatzmanParameterTestIdealOfCMRings}) it is shown that
there is a surjection $\Ann_E I \rightarrow \HH^2_{\mathfrak{m}} (R/I)$
which is compatible with with the induced $p^1$-linear map on $\HH^2_{\mathfrak{m}} (R/I)$
and the $p^1$-linear map $u T$ on $\Ann_E I$ with
$$u=x_1^3x_2x_3 + x_1^3x_2x_4+x_1^2x_3x_4x_5+ x_1x_2x_3x_4x_5+ x_1x_2x_4^2x_5+ x_2^2x_4^2x_5+x_3x_4^2x_5^2+ x_4^3x_5^2 .$$


\begin{verbatim}
i2 :      R=ZZ/2[x_1..x_5];
i3 :      I=minors(2, matrix {{x_1,x_2,x_2,x_5},{x_4,x_4,x_3,x_1}} )
                                                   2
o3 = ideal (x x  + x x , x x  + x x , x x  + x x , x  + x x , x x  + x x , x x
             1 4    2 4   1 3    2 4   2 3    2 4   1    4 5   1 2    4 5   1 2
     --------------------------------------------------------------------------
     + x x )
        3 5
o3 : Ideal of R
i4 :      u=x_1^3*x_2*x_3 + x_1^3*x_2*x_4+x_1^2*x_3*x_4*x_5+ x_1*x_2*x_3*x_4*x_5+
          x_1*x_2*x_4^2*x_5+ x_2^2*x_4^2*x_5+x_3*x_4^2*x_5^2+ x_4^3*x_5^2;
i5 :      compatibleIdeals(u)
             3        3        2                           2      2 2
o5 = {ideal(x x x  + x x x  + x x x x  + x x x x x  + x x x x  + x x x  +
             1 2 3    1 2 4    1 3 4 5    1 2 3 4 5    1 2 4 5    2 4 5
     --------------------------------------------------------------------------
        2 2    3 2                    2
     x x x  + x x ), ideal (x  + x , x  + x x ), ideal (x , x ), ideal (x , x ,
      3 4 5    4 5           1    2   1    4 5           4   1           4   1
     --------------------------------------------------------------------------

     x ), ideal (x , x , x , x ), ideal (x , x , x , x , x ), ideal (x , x ,
      5           5   4   1   3           5   4   3   2   1           5   4
     --------------------------------------------------------------------------

     x , x ), ideal (x , x , x ), ideal (x , x , x , x ), ideal (x , x , x ),
      2   1           4   1   3           4   3   2   1           4   2   1
     --------------------------------------------------------------------------
                               2
     ideal (x  + x , x  + x , x  + x x ), ideal (x  + x , x , x , x ), ideal
             3    4   1    2   2    4 5           3    4   2   1   5
     --------------------------------------------------------------------------
     (x , x , x )}
       2   1   5

o5 : List
\end{verbatim}

\section{Future plans}

\bibliographystyle{skalpha}
\bibliography{MainBib}



\end{document}
