\documentclass{amsart}
\usepackage{calc,amssymb,amsthm,amsmath,amsfonts,setspace,mathtools}
\RequirePackage[dvipsnames,usenames]{xcolor}
\usepackage{textgreek}
\usepackage[utf8x]{inputenc}
\DeclareUnicodeCharacter{957}{\textnu}

\usepackage{microtype}

\usepackage{hyperref} \hypersetup{ bookmarks, bookmarksdepth=3,
   bookmarksopen, bookmarksnumbered, pdfstartview=FitH,
   colorlinks,backref,hyperindex, linkcolor=Sepia,
   anchorcolor=BurntOrange, citecolor=MidnightBlue,
   citecolor=OliveGreen, filecolor=BlueViolet, menucolor=Yellow,
   urlcolor=OliveGreen } \usepackage{xspace}
\interfootnotelinepenalty=100000

\usepackage{mabliautoref}
\usepackage{colonequals}
\frenchspacing
%\input{kmacros3.sty}

\setlength{\parskip}{0.4em}

\usepackage{verbatim}
\usepackage{enumerate}
%\usepackage[normalem]{ulem}
%\usepackage{marginnote}
%\newcommand{\fram}{\mathfrak{m}}

\usepackage{fancyvrb}
\DefineVerbatimEnvironment%
{MyVerbatim}{Verbatim}
{formatcom=\color{Violet}}

\usepackage{fancyhdr}

\pagestyle{fancy}
\fancyhead[RO, LE]{\thepage}
\fancyhead[C]{The \emph{FrobeniusThresholds} package for \emph{Macaulay2}}
\fancyfoot{}
\renewcommand{\headrulewidth}{0pt}

\makeatletter
\def\@settitle{\begin{center}%
  \baselineskip14\p@\relax
    \bfseries
    \normalfont\LARGE%<- NEW
\@title
  \end{center}%
}
\makeatother

\newcommand{\ft}{\operatorname{c}}
\newcommand{\idealm}{\mathfrak{m}}
\DeclareMathOperator{\fpt}{fpt}

\renewcommand{\geq}{\geqslant}
\renewcommand{\leq}{\leqslant}
\renewcommand{\ge}{\geqslant}
\renewcommand{\le}{\leqslant}

%%%%%%%%%%%%%%%%%% To-Do %%%%%%%%%%%%%%%%%%
\usepackage[textwidth=3.3 cm,textsize=small,shadow
%disable
%%option disable removes the notes
]{todonotes}

\newcommand{\warning}[2][] 
{\todo[color=red,caption={},#1]{#2}} 
\newcommand{\postit}[2][] 
{\todo[linecolor=orange,backgroundcolor=yellow!40, caption={},#1]{#2}} 
\newcommand{\daniel}[2][] 
{\todo[linecolor=green,backgroundcolor=green!10,caption={}, #1]{#2}} 
\newcommand{\pedro}[2][] 
{\todo[linecolor=blue,backgroundcolor=blue!10,caption={}, #1]{#2}} 
\newcommand{\emily}[2][] 
{\todo[linecolor=gray,backgroundcolor=gray!20,caption={}, #1]{#2}} 
\newcommand{\karl}[2][] 
{\todo[linecolor=niceblue,backgroundcolor=niceblue!20,caption={}, #1]{#2}} 


\begin{document}
\title{Response to referee's comments}
\author{Daniel J.\ Hern\'andez,  Karl Schwede, Pedro Teixeira, and Emily E.\ Witt}
\maketitle

\begin{enumerate}
\item We have added a footnote pointing out the previous name of \emph{FrobeniusThresholds}.  Before submitting the current paper, we contacted each author to decide who should be an official author/contributor for the current paper and for the \emph{FrobeniusThresholds} package.
Though the package \emph{FrobeniusThresholds} broke off from the \emph{TestIdeals} package, not all authors of the \emph{TestIdeals} package contributed to \emph{FrobeniusThresholds}.
The \emph{Macaulay2} site for \emph{FrobeniusThresholds}, which previously listed all the \emph{TestIdeals} authors, was recently updated, and now has the correct list of authors and contributors for the package. 
We now thank all \emph{TestIdeals} authors in the Acknowledgments. 
\item We do think that this paragraph is important, since $F$-thresholds and $F$-pure thresholds are the focus of the package, and they are constructed in this way.  We've added another example of the Hilbert--Kunz multiplicity to better show that this type of construction indeed fits into a broader context. 
\item Though we don't particularly like it either, this is standard notation, initiated in [MTW05].
\item We have added a brief explanation about why these limits exist. 
\item We changed the wording to reflect the fact that it is remarkable that the terms of the sequence can be recovered by its limit. 
\item We updated this based on the referee's suggestion. 
\item We moved this before Subsection 2.1.  We also reorganized 2.1 to highlight options that might be more relevant to the user. 
\item Since this computation generally sits in the \emph{TestIdeals} package, we thought it better to avoid the  $\epsilon$ notation and technical description of how these functions are used. 
\item ``Clearly'' is removed, and an example is added to motivate this remark. 
\item We've put permalinks to the cited versions (the current ones) of our packages  \emph{FrobeniusThresholds} and \emph{TestIdeals}. 
However, we will need to consult with the editors about how to deal with citing other packages.  
\item We investigated what is going on here, and it appears that the way to avoid the error is to first install the most recent version of the \emph{TestIdeals} package from GitHub first, and then install the current version of \emph{FrobeniusThresholds}. 
This issue will not exist after the next version of \emph{Macaulay2} is released.

\end{enumerate}


\end{document}
